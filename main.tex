\documentclass{article}
\usepackage[utf8]{inputenc}
\usepackage[english]{babel}
\usepackage{amsthm,amssymb,amsmath}
\usepackage{titlesec}
\usepackage{url}

\newtheorem*{theorem}{Theorem}

\newcommand{\NN}{\mathbb{N}}
\newcommand{\ZZ}{\mathbb{Z}}
\newcommand{\RR}{\mathbb{R}}
\newcommand{\QQ}{\mathbb{Q}}
\newcommand{\CC}{\mathbb{C}}

\title{\textbf{Simplicial Homology} \\[0.2cm] \large An introduction to simplicial homology theory}

\author{Luciano Melodia}
\date{\today}

\begin{document}

\maketitle
\begin{abstract}
This paper examines the fundamental ideas of simplicial structures that lead to simplicial homology theory and introduces singular homology to demonstrate the equivalence of homology groups of homomorphic topological spaces. It concludes with a proof of the equivalence of simplicial and singular homology groups.
\end{abstract}
\tableofcontents

\section{Simplicial Complexes}
\section{Homology Groups}
\section{Singular Homology}
\section{Chain Complexes}
\section{Exact Sequences}
\section{Relative Homology Groups}
\section{The Equivalence of $H_k^\Delta$ and $H_k$}

\bibliographystyle{apa}
\begin{thebibliography}{100}
\bibitem{1} Edelsbrunner, H., Harer, J. L. (2022). Computational Topology: An Introduction. American Mathematical Society.
\bibitem{2} Hatcher, A. (2005). Algebraic Topology. Cambridge University Press.
\bibitem{3} Jonsson, J. (2011). Introduction to Simplicial Homology. Königliche Technische Hochschule. URL: \url{https://people.kth.se/~jakobj/doc/homology/homology.pdf}.
\bibitem{4} Khoury, M. (2022). Lecture 6: Introduction to Simplicial Homology. Topics in Computational Topology: An Algorithmic View. Ohio State University. URL: \url{http://web.cse.ohio-state.edu/~wang.1016/courses/788/Lecs/lec6-marc.pdf}.
\bibitem{5} Nadathur, P. (2007). An Introduction to Homology. University of Chicago. URL: \url{https://www.math.uchicago.edu/~may/VIGRE/VIGRE2007/REUPapers/FINALFULL/Nadathur.pdf}.

\end{thebibliography}
\end{document}
