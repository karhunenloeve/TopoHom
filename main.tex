\documentclass{amsart}
\usepackage{amsaddr}
\usepackage[utf8]{inputenc}
\usepackage[english]{babel}
\usepackage{amsthm,amssymb,amsmath}
\usepackage{tikz-cd}
\usepackage{url}
\usepackage[backend=biber]{biblatex}
\newtheorem{theorem}[section]{Theorem}
\newtheorem{definition}{Definition}[section]
\newtheorem{lemma}[definition]{Lemma}
\newtheorem{proposition}[definition]{Proposition}
\newtheorem{corollary}[definition]{Corollary}
\newtheorem{example}[definition]{Example}

\newcommand{\NN}{\mathbb{N}}
\newcommand{\ZZ}{\mathbb{Z}}
\newcommand{\RR}{\mathbb{R}}
\newcommand{\QQ}{\mathbb{Q}}
\newcommand{\CC}{\mathbb{C}}

\title{Notes on Simplicial and Singular Homology}

\author[Luciano Melodia]{Luciano Melodia}
\address{Department of Mathematics \\ Friedrich-Alexander Universität Erlangen-Nürnberg \\ Cauerstraße 11 \\ 91058 Erlangen}
\email{luciano.melodia@fau.de}
\subjclass[2020]{55N10, 55N99}
\date{\today}

\begin{document}
\maketitle

\begin{abstract}
This paper explores the foundational concepts of simplicial structures that form the basis of simplicial homology theory. It also introduces singular homology as a means to establish the equivalence of homology groups for homeomorphic topological spaces. The paper concludes by providing a proof of the equivalence between simplicial and singular homology groups.

We follow the structure and explanations provided by Nadathur [7] and Hatcher [3]. In particular, the definitions are taken from the introductory book of Boissonnat et al. [1], as well as from Jonsson's introduction [4] and the paper on computational topology by Melodia et al. [6]. Some individual lemmas and proof ideas are drawn from Khoury [5] or from the textbook by Edelsbrunner et al. [2], but they have been adapted, expanded, and implemented independently. To enhance readability, we have omitted reference citations within the text.
\end{abstract}

\tableofcontents

\section{Simplicial Complexes}
We would like to emphasize that a collection of points $X = \{x_0, x_1, \ldots, x_d\}$ in $\mathbb{R}^n$ is considered to be \emph{affinely independent} if these points do not lie within any affine subspace of dimension lower than $d$.

\begin{definition}
Given a set $X = \{x_0, x_1, \ldots, x_d\} \subset \mathbb{R}^n$ consisting of $d+1$ \textbf{affinely independent} points, the $d$-dimensional simplex $\sigma^{(d)}$, also known as a \textbf{\emph{$d$-simplex}}, is defined as the set of all convex combinations of these points.
\begin{equation}
	\sigma^{(d)} := \left\{\sum_{i=0}^{d} \lambda_i x_i \; \vert \; \sum_{i=0}^{d} \lambda_i = 1, \; \lambda_i \geq 0 \right\}.
\end{equation}
\end{definition}

As a convention, the empty set $\emptyset$ is included as a face, representing the simplex formed by the empty subset of vertices. A $0$-simplex represents a single point, a $1$-simplex represents a line segment connecting two points, a $2$-simplex represents a triangle, and a $3$-simplex represents a tetrahedron. It is worth mentioning that the $d$-simplex is homeomorphic to the $d$-dimensional disk $D^d$.

Furthermore, it is worth noting that $\sigma^{(d)}$ represents the convex hull of the points $X$, which can be defined as the smallest convex subset of $\mathbb{R}^n$ that contains all the points $x_0, x_1, \ldots, x_d$. The \emph{faces} of the simplex $\sigma^{(d)}$ with vertex set $X$ are simplices formed by subsets of $X$. An \emph{$d$-face} of a simplex refers to a subset of the vertices of the simplex with a cardinality of $d+1$. The faces of an $d$-simplex with a dimension less than $d$ are known as its \emph{proper faces}. Two simplices are considered to be \emph{properly situated} if their intersection is either empty or a face of both simplices. By identifying simplices along entire faces, we can construct the resulting \emph{simplicial complexes}.

\begin{definition}
A \textbf{simplicial complex} $K$ is a finite collection of simplices that satisfies the following properties:
\begin{enumerate}
	\item For every simplex $\sigma^{(d)}$ in $K$ and every face $\tau^{(k)}$ with $k < d$ of $\sigma^{(d)}$, it follows that $\tau^{(k)}$ is also in $K$.
	\item If $\sigma^{(d)}$ and $\tau^{(k)}$ are both simplices in $K$, then they are properly situated.
\end{enumerate}
\end{definition}

The \emph{dimension} of $K$ is defined as the highest dimension among its simplices. For a simplicial complex $K$ in $\mathbb{R}^n$, its \emph{underlying space} $\vert K \vert$ is the union of all the simplices in $K$. The topology of $K$ is determined by the topology induced on $\vert K \vert$ by the standard topology in $\mathbb{R}^n$. It is important to note that when the vertex set is known, a simplicial complex in $\mathbb{R}^n$ can be fully characterized by listing its simplices. As a result, we can describe it purely in terms of combinatorics using \emph{abstract simplicial complexes}.

\begin{definition}
Consider a finite set $V = \{v_1, \ldots, v_n\}$. An \textbf{abstract simplicial complex $\tilde{K}$} with vertex set $V$ is a collection of finite subsets of $V$ that satisfies the following two conditions:
\begin{enumerate}
	\item All elements of $V$ are included in $\tilde{K}$.
	\item If $\sigma^{(d)}$ is a subset of $\tilde{K}$ and $\tau^{(k)}$ is a subset of $\sigma^{(d)}$, then $\tau^{(k)}$ is also a subset of $\tilde{K}$.
\end{enumerate}
\end{definition}

The abstract simplicial complex $\tilde{K}$ associated with a simplicial complex $K$ is commonly referred to as its \emph{vertex scheme}. Conversely, if an abstract complex $\tilde{K}$ serves as the vertex scheme for a complex $K$ in $\mathbb{R}^n$, then $K$ is known as a \emph{geometric realization} of $\tilde{K}$.

\begin{lemma}
Every finite abstract simplicial complex $\tilde{K}$ can be realized geometrically in a Euclidean space.
\end{lemma}

\begin{proof}
Let $\{v_1,v_2, \ldots, v_n\}$ denote the vertex set of $\tilde{K}$, where $n$ represents the number of vertices in $\tilde{K}$. Consider $\sigma^{(n-1)} \subset \mathbb{R}^n$, the simplex formed by the span of $\{e_1, e_2, \ldots, e_n\}$, where $e_i$ represents the $i$th unit vector. In this context, $K$ refers to the subcomplex of $\sigma^{(d)}$ such that $[e_{i_0}, \ldots, e_{i_d}]$ is a $d$-simplex of $K$ if and only if $[v_{i_0}, \ldots, v_{i_d}]$ is a simplex of $\tilde{K}$.
\end{proof}

\paragraph{Note:}
All realizations of an abstract simplicial complex are homeomorphic to each other. The specific realization mentioned above is referred to as the \emph{natural realization}. Furthermore, it has been proven that any finite abstract simplicial complex of dimension $d$ can be realized as a simplicial complex in $\mathbb{R}^{2d+1}$.

\section{Homology Groups}
Given a set $V$ representing the vertices of a $d$-simplex $\sigma^{(d)}$, we can establish an \emph{orientation} for the simplex by selecting a specific ordering for the vertices. If the vertex ordering differs from our chosen order by an odd permutation, it is considered \emph{reversed}, while even permutations are said to \emph{preserve} the orientation. Consequently, any simplex can have only two possible orientations. Moreover, the orientation of a $d$-simplex induces an orientation on its $(d-1)$-faces. To be more precise, if $\sigma^{(d)} := (v_0, v_1, \ldots, v_d)$ represents an oriented $d$-simplex, then the orientation of the $(d-1)$-face $\tau^{(d-1)}$ of $\sigma^{(d)}$ with the vertex set $\{v_0,\ldots,v_{i-1},v_{i+1},\ldots,v_d\}$ is given by $\tau_i^{(d-1)} = (-1)^i (v_0, \ldots,v_{i-1},v_{i+1},\ldots,v_d)$.

\begin{definition}
Given a set $\{\sigma_1^{(d)}, \ldots, \sigma_k^{(d)}\}$ of arbitrarily oriented $d$-simplices of a complex $K$ and an abelian group $G$, we define a \textbf{$d$-chain} $c$ with coefficients $g_i \in G$ as a formal sum.
\begin{equation}
c := g_1 \sigma^{(d)}_1 + g_2 \sigma^{(d)}_2 + \ldots + g_k \sigma^{(d)}_k = \sum_{i=1}^{k} g_i \sigma^{(d)}_i.
\end{equation}
\end{definition}

\paragraph{Note:} Henceforth we will assume that $G = (\mathbb{Z},+)$.

\begin{lemma}
The set of simplicial $d$-chains $L^\Delta_d$ is an abelian group $(L^\Delta_d,+)$.
\end{lemma}
\begin{proof}
The identity element of the group is represented by the empty chain $\sum_{i=1}^{k}$ $e_G \sigma^{d}_i$ $= e_{L_d^\Delta}$. The sum of two chains is defined as $c+c' = \sum_{i=1}^{k} g_i \sigma_i^{(d)}$ $+ \sum_{j=1}^{l} g'_j \sigma_j^{(d)}$ $=$ $\sum_{i=1}^{k} (g_i+g_i') \sigma_i^{(d)}$ $+$ $\sum_{j=k}^{l} g'_j \sigma_j^{(d)}$ if $k \leq l$ and $c+c' = \sum_{i=1}^{k} g_i \sigma_i^{(d)} + \sum_{j=1}^{l} g'_j \sigma_j^{(d)} = \sum_{i=1}^{l} (g_i+g_i') \sigma_i^{(d)} + \sum_{j=l}^{k} g_j \sigma_j^{(d)}$ if $k > l$, thus, we can conclude that $c+c' \in L^\Delta_d$. The associativity of the group operation in $L^\Delta_d$ follows directly from the associativity of the group operation in $G$. The inverse element is defined by $e_{L^\Delta_d} = c + (-c) = \sum_{i=1}^{k} g_i \sigma_i^{(d)} - \sum_{i=1}^{k} (-g_i) \sigma_i^{(d)} = \sum_{i=1}^{k} (g_i-g_i) \sigma_i^{(d)}$ with $c,-c \in L^\Delta_d$.
\end{proof}

\begin{definition}
Let $\sigma^{(d)}$ be an oriented $d$-simplex in a complex $K$. The $\textbf{boundary}$ of $\sigma^{(d)}$ is defined as the simplicial $(d-1)$-chain of $K$ with coefficients in the abelian group $g_i \in G = \mathbb{Z}$, given by
\begin{equation}
\partial(\sigma^{(d)}) = g_0 \sigma^{(d-1)}_0 + g_1 \sigma^{(d-1)}_1 + \ldots + g_d \sigma^{(d-1)}_d = \sum_{i=1}^{d} g_i \sigma^{(d-1)}_i
\end{equation}
where $\sigma^{(d-1)}_i$ is an $(d-1)$-face of $\sigma^{(d)}$. If $d=0$, we define $\partial(\sigma^{(0)}) = e_G = 0$.
\end{definition}

Since $\sigma^{(d)}$ is an oriented simplex, the $\sigma^{(d-1)}_i$-faces also have associated orientations. We can extend the definition of the boundary linearly to elements of $L^\Delta_d$.

\begin{lemma}
The \textbf{boundary operator} is a group homomorphism $\partial: L^\Delta_d \rightarrow L^\Delta_{d-1}$.
\end{lemma}
\begin{proof}
We define the boundary operator for a $d$-chain $c = \sum_{i=1}^{k} g_i \sigma_i^{(d)}$ as follows: $\partial(c) = \sum_{i=1}^{k} g_i \partial(\sigma_i^{(d)}) = \sum_{i=1}^{k} g_i \sum_{j=1}^{d} \sigma_j^{(d-1)} = \sum_{i=1}^{k} \sum_{j=1}^{d} g_i \sigma_j^{(d-1)} \in L^\Delta_{d-1}$, where $\sigma_i^{(d)}$ are the $d$-simplices of $K$. We can compute this by
\begin{align}
\partial(c + c') &= \partial(\sum_{i=1}^{k} g_i \sigma_i^{(d)} + \sum_{j=1}^{l} g'_j \sigma_j^{(d)}) \\
&= \sum_{i=1}^{k} g_i \partial(\sigma_i^{(d)}) + \sum_{j=1}^{l} g'_j \partial(\sigma_j^{(d)}) \\
&= \partial(c) + \partial(c').
\end{align}
\end{proof}

\begin{example}
Let's consider the $2$-simplex $\sigma^{(2)}$ with vertices $v_0$, $v_1$, and $v_2$. The $1$-faces of this simplex are $e_0 = (v_1,v_2)$ connecting $v_1$ and $v_2$, $e_1 = (v_2,v_0)$ connecting $v_2$ and $v_0$, and $e_2 = (v_0,v_1)$ connecting $v_0$ and $v_1$. Now, let's proceed with the computation.
\begin{align}
\partial(\partial(\sigma^{(2)})) &= \partial (e_1+e_2+e_3) \\
&= \partial e_1 + \partial e_2 + \partial e_3 \\
&= \partial(v_0,v_1) + \partial(v_1,v_2) + \partial(v_2,v_0) \\
&= [(v_1)-(v_0)] + [(v_2)-(v_1)]+[(v_0)-(v_2)].
\end{align}
We observe that $L^\Delta_0$ is an abelian group and that oppositely oriented simplices cancel each other out, resulting in $\partial(\partial(\sigma^{(2)})) = 0$. This property can be generalized to higher dimensions through induction. Therefore, since $\partial$ is a linear operator and the chain $c$ is a sum of $d$-simplices, we can conclude that $\partial^2(c) = 0$ for any $d$-chain $c$ in $L^\Delta_d$. Consequently, the boundary of the boundary is zero. Moreover, if the boundary of a simplex is zero, it is referred to as a \emph{cycle}. By this definition, we can deduce that the boundary of any simplex is a cycle.
\end{example}

\begin{definition}
A $d$-chain is referred to as a \textbf{cycle} if its boundary is equal to zero. We denote the set of $d$-cycles of a complex $K$ over the group $\mathbb{Z}$ as $Z_d$, the simplicial \textbf{cycle group}. It is important to note that $Z^\Delta_d$ is a subgroup of $L^\Delta_d$ and can also be expressed as $Z^\Delta_d = \ker(\partial_d)$.
\end{definition}

\begin{definition}
A $d$-cycle of a $k$-complex $K$ is said to be \textbf{homologous to zero} if it can be expressed as the boundary of an $(d+1)$-chain in $K$, where $d=0,1,\ldots,k-1$. In other words, a cycle is considered a boundary if it can be \glqq filled in\grqq{} by a higher-dimensional chain. This equivalence relation is denoted as $c \sim 0$, and the subgroup of $Z^\Delta_d$ consisting of boundaries is referred to as the simplicial \textbf{boundary group $B^\Delta_d$}. It is worth noting that $B^\Delta_d$ is equal to the image of the boundary operator $\partial_{d+1}$.
\end{definition}

Since $B^\Delta_d$ is a subgroup of $Z^\Delta_d$ and $Z^\Delta_d$ is an abelian group, every subgroup of $Z^\Delta_d$ is normal. Therefore, we can construct the quotient group $H^\Delta_d = Z^\Delta_d / B^\Delta_d$.

\begin{definition}
The group $H^\Delta_d$ represents the $d$-dimensional simplicial \textbf{homology group} of the complex $K$ over $\mathbb{Z}$. It can be expressed as the quotient group $\ker(\partial_d) / \text{im}(\partial_{d+1})$.
\end{definition}

Next, we want to examine the structure of this homology group by shedding light on its connection to the connected components of a simplicial complex. We will find that the homology groups of the connected components of the complex, which in turn form a complex themselves, yield the direct sum of the homology group of the entire complex.

\begin{definition}
A \textbf{subcomplex} is defined as a subset $S$ of the simplices belonging to a complex $K$, where $S$ itself forms a complex.
\end{definition}

\begin{definition}
The collection of all simplices in a complex $K$ with dimensions less than or equal to $d$ is referred to as the \textbf{$d$-skeleton} of $K$.
\end{definition}

By definition, the $d$-skeleton forms a subcomplex.

\begin{definition}
A complex $K$ is considered \textbf{connected} if it cannot be expressed as the disjoint union of two or more non-empty subcomplexes. A geometric complex is \textbf{path-connected} if there exists a path consisting of $1$-simplices connecting any vertex to any other vertex.
\end{definition}

\begin{lemma}
\label{pathconnect}
Path-connectedness $\Longleftrightarrow$ connectedness.
\end{lemma}

\begin{proof}
\glqq $\Longrightarrow$\grqq{}: Let us assume that $K$ is not connected. In this case, we can choose two separate subcomplexes, namely $L$ and $M$, which do not share any common elements, but when combined, they form the entire complex $L \cup M = K$. Now, let's suppose that there exists a path between a vertex $l_0$ in $L$ and a vertex $m_0$ in $M$. However, if we consider the last vertex $l_i$ in this path that belongs to $L$, we observe that the $1$-simplex connecting $l_i$ to the next vertex in the path cannot be a part of either $L$ or $M$. If it were, then $L$ and $M$ would have a nonempty intersection, which contradicts our initial assumption that $K$ is not connected.

\glqq $\Longleftarrow$\grqq{}: Now, let's consider the other direction. Suppose there are two points, namely $l_0$ and $m_0$, in $K$ that do not have a path connecting them. In this case, we can define $L$ as the path-connected subcomplex of $K$ that contains $l_0$, and $M$ as the path-connected subcomplex that contains $m_0$. If there exists a vertex $v_0$ in the intersection of $L$ and $M$ (i.e., $v_0 \in L \cap M \neq \emptyset$), then we can find a path from $l_0$ to $v_0$ and another path from $v_0$ to $m_0$. By concatenating these paths, we obtain a path from $l_0$ to $m_0$, which contradicts our initial assumption that there is no path between $l_0$ and $m_0$. Therefore, we conclude that $L$ and $M$ must have an empty intersection ($L\cap M= \emptyset$), indicating that $K$ is not connected.
\end{proof}

\begin{theorem}
\label{decomptheorem}
Let $K_1, \ldots, K_p$ be the collection of all connected components of a complex $K$. Furthermore, let $H^\Delta_{d_i}$ represent the $d$th simplicial homology group of $K_i$, and $H^\Delta_d$ denote the $d$th simplicial homology group of $K$. In this context, we can establish that $H^\Delta_d$ is isomorphic to the direct sum $H^\Delta_{d_1} \oplus \cdots \oplus H^\Delta_{d_p}$.
\end{theorem}

\begin{proof}
Let $L^\Delta_d$ represent the group of simplicial $d$-chains of $K$, and $K_i$ denote the $i$th component of $K$. We can define $L^\Delta_{d_i}$ as the group of simplicial $d$-chains of $K_i$. It is evident that $L^\Delta_{d_i}$ is a subgroup of $L^\Delta_d$. Furthermore, we observe that $L^\Delta_d$ can be expressed as the direct sum of $L^\Delta_{d_1}, \ldots, L^\Delta_{d_p}$:
\begin{equation}
L^\Delta_d = L^\Delta_{d_1} \oplus \cdots \oplus L^\Delta_{d_p}.
\end{equation}
Our goal is to demonstrate that a similar decomposition can be applied to the groups $B^\Delta_d$ and $Z^\Delta_d$. By considering $B^\Delta_{d_i}$ as the image of $\partial_{d+1}$ restricted to the subgroup $L^\Delta_{d_i}$, we can represent the group $B^\Delta_d$ as the direct sum of these restrictions:
\begin{equation}
B^\Delta_d = B^\Delta_{d_1} \oplus \cdots \oplus B^\Delta_{d_p}.
\end{equation}
Thus, for any element $c \in L^\Delta_{d+1}$, which can be represented as:
\begin{align}
c = c_1 + \cdots + c_p, \quad \partial_{d+1}(c) = \partial_{d+1}(c_1) + \cdots + \partial_{d+1}(c_p) \in B^\Delta_d,
\end{align}
where $c_i \in L^\Delta_{{(d+1)}_i}$.
Let us define $Z^\Delta_{d_i}$ as the intersection of the kernel of $\partial_{d}$ and $L^\Delta_{d_i}$. It follows that $Z^\Delta_d$ can be expressed as the direct sum of $Z^\Delta_{d_1}, \ldots, Z^\Delta_{d_p}$:
\begin{equation}
Z^\Delta_d = Z^\Delta_{d_1} \oplus \cdots \oplus Z^\Delta_{d_p}.
\end{equation}
To verify this, we observe that for an element $c \in L^\Delta_d$ to belong to $Z^\Delta_d$, we require $\partial_{d}(c) = 0$. However, we can express $\partial_{d}(c)$ as $\partial_{d}(c_1) + \cdots + \partial_{d}(c_p)$. Therefore, for $\partial_{d}(c) = 0$ to hold, it implies that $\partial_{d}(c_i) = 0$, indicating that $c_i \in Z^\Delta_{d_i}$.
Since both $Z^\Delta_d$ and $B^\Delta_d$ can be decomposed componentwise, we can conclude that:
\begin{equation}
Z^\Delta_d / B^\Delta_d = Z^\Delta_{d_1} / B^\Delta_{d_1} \oplus \cdots \oplus Z^\Delta_{d_p} / B^\Delta_{d_p},
\end{equation}
and consequently:
\begin{equation}
H^\Delta_d = H^\Delta_{d_1} \oplus \cdots \oplus H^\Delta_{d_p}.
\end{equation}
\end{proof}

\begin{definition}
The \textbf{index} of a chain $c = \sum_{i=1}^{k} g_i \sigma_i^{(d)}$ is defined as the sum of the coefficients $I(c) = \sum_{i=1}^{k} g_i$.
\end{definition}

\begin{proposition}
\label{decomposition}
If $K$ is a connected complex and $c$ is a $0$-chain with $I(c) = 0$, then the condition $I(c) = 0$ is equivalent to $c \sim 0$, where $\sim$ denotes homology equivalence. Furthermore, in this case, the zeroth simplicial homology group $H^\Delta_0(K,\mathbb{Z})$ is isomorphic to the integers $\mathbb{Z}$.
\end{proposition}

\begin{proof}
We begin by proving that $c \sim 0 \implies I(c) = 0$. Let $\sigma^{(1)} = (v_0,v_1)$ be a $1$-simplex. Then, for a chain $c = \partial_1(g\sigma^{(1)}) = gv_1-gv_0$, we have $c \sim 0$. It is clear that $I(c) = I(g\sigma^{(1)}) = g-g = 0$. Since $I(c+c') = I(c) + I(c')$, $I$ is a group homomorphism. For any $c \in L^\Delta_1$ of the form $\sum_{i=1}^{k} g_i \sigma_i^{(1)}$, where $\sigma_i^{(1)} = (v_i,v_{i+1})$, we have $c = \partial_1(c) \sim 0 \implies I(c) = I(\partial_1(c)) = 0$.

To prove the forward direction, $I(x) = 0 \implies c \sim 0$, we consider two vertices $v$ and $w$ of $K$. Since $K$ is connected, there exists a path between $v$ and $w$ consisting of $1$-simplices $\sigma_i^{(1)} = (v_i,v_{i+1})$, $i=1,\ldots,k-1$, where $v_0 = v$ and $v_k = w$. We consider the boundary of the chain $c = \sum_{i=1}^{k} g \sigma_i^{(1)}$, given by $\partial_1(c) = \sum_{i=1}^{k}g \partial_1(\sigma_i^{(1)}) = \sum_{i=1}^{k}g [(v_{i+1}) - (v_i))] = gw - gv$. Since $\partial_1(c)$ is a boundary, we have $c = \partial_1(c) \sim 0$. This implies that $(gw-gv) \sim 0$, which further implies $gw \sim gv$. Therefore, any $0$-chain $c$ in $K$ is homologous to the chain $gv$. We observe that homologous chains have equal indices, i.e., $I(c) = I(gv) = g$. Thus, we have $c \sim gv \implies c \sim I(c)v$. This shows that if $I(c) = 0$, then $c \sim 0$. Hence, $I(c) = 0$ is equivalent to $c \sim 0$.

As mentioned, $I$ is a homomorphism from $L^\Delta_0 = Z^\Delta_0$ to $\mathbb{Z}$. For a $0$-simplex $c$ and $g \in \mathbb{Z}$, the chain $gc \in L^\Delta_0$ is a cycle with $I(gc) = g$. Therefore, $I(Z^\Delta_0) = \mathbb{Z}$. Since $I(c) = 0$ is equivalent to $c \sim 0$, we have $B^\Delta_0 = \ker(I)$. This implies that $H^\Delta_0 = Z^\Delta_0/B^\Delta_0 \cong \mathbb{Z}$.
\end{proof}

At this point, we can deduce the following corollary from Theorem \ref{pathconnect} and Proposition \ref{decomposition}:

\begin{corollary}
\label{directsum0hom}
The zero-dimensional simplicial homology group of a complex $K$ over $\mathbb{Z}$ can be represented as $\mathbb{Z}^p = \bigoplus_p \mathbb{Z}$, where $p$ denotes the number of connected components present in $K$.
\end{corollary}

\begin{example}
\begin{itemize}
	\item[]
	\item This implies that the zeroth homology group of the circle is isomorphic to $\mathbb{Z}$. If we consider a simplicial representation of the circle using four one-simplices, $v_1 = (w,v)$, $v_2 = (v,y)$, $v_3 = (y,x)$, and $v_4 = (x,w)$, the group $Z_0^\Delta$ consists of sums over the four zero-simplices $v$, $w$, $x$, and $y$ with coefficients in $\mathbb{Z}$. Let $c$ be a zero-chain with non-zero coefficients given by
	\begin{equation}
	c = g_1v+g_2w+g_3x+g_4y.
	\end{equation}
	In order to reduce it to an element of $H^\Delta_0$, we subtract from it the chain $c' = g_4x-g_4y \sim 0$, resulting in
	\begin{equation}
	c-c' = g_1v+g_2w +(g_3-g_4)x.
	\end{equation}
	By repeating this process, we obtain a new chain
	\begin{equation}
	c'' = (g_1-g_2+g_3-g_4)v.
	\end{equation}
	Since $c'' \sim c$, it represents an element of $H^\Delta_0$. Moreover, since $g_i \in \mathbb{Z}$, we can write $(g_1-g_2+g_3-g_4) \in \mathbb{Z}$ as $c'' = gv$, where $g$ is an element of $\mathbb{Z}$. Therefore, we can choose any $g$, which implies that $H^\Delta_0 \cong \mathbb{Z}$.
	\item We will demonstrate that $H^\Delta_d(S^d) \cong \mathbb{Z}$. It is worth noting that the $d$-simplex $\sigma^{(d)}$ and the $d$-ball are homeomorphic. Consequently, their boundaries, which consist of $(d-1)$-simplices, and the $d$-sphere are also homeomorphic. Therefore, the appropriate simplicial structure to impose on $S^d$ is that of the boundary of the $(d+1)$-simplex $\sigma^{(d+1)}$. Let ${v_0,\ldots,v_{d-1}}$ denote the set of vertices of $\sigma^{(d+1)}$. It is important to note that this set is not oriented, and the orientations of the $(d-1)$-simplices can be arbitrarily determined. We will utilize their numbering to establish orientations. Consequently, all $d$-chains on this structure can be expressed as:
	\begin{equation}
	\label{chain}
	c = \sum_{i=0}^{d+1}g_i (v_0,\ldots,v_{i-1},v_{i+1},\ldots,v_d),
	\end{equation}
	where $g_i \in \mathbb{Z}$. Since $\sigma^{(d+1)}$ itself is not part of the structure, there are no boundaries in $Z^\Delta_d$, the group of simplicial cycles. Therefore, $H^\Delta_d = Z\Delta_d/B^\Delta_d$ represents the group of simplicial cycles. If $c \in Z^\Delta_d$, then $\partial_{d+1}(c) = 0$. By using Eq. \ref{chain}, we have:
	\begin{align} \partial_{d+1}(c) = &\partial_{d+1}\left(\sum_{i=0}^{d+1} g_i(v_0,\ldots,v_{i-1},v_{i+1},\ldots,v_d) \right) \\ = &\sum_{i=0}^{d+1}g_i \big(\sum_{j1}^{d+1}(-1)^j(v_0,\ldots,v_{i-1},v_{i+1},\ldots,v_{j-1},v_{j+1},\ldots,v_d)\big). \end{align}
	By rearranging this sum, we obtain terms of the form:
	\begin{equation} \label{terms} (g_k-g_l)(v_0,\ldots,v_{j-1},v_{j+1},\ldots,v_{i-1},v_{i+1},\ldots,v_d) \end{equation}
	where $k,l = 0, \ldots, d+1$ for all $i,j = 0, \ldots, d$. Each pair of $d$-simplices of $\sigma^{(d+1)}$ intersect along an $(d-1)$-face. Therefore, we obtain terms of the form given in Eq. \ref{terms} for each of these faces. From this, we can deduce that if $\partial_{d}(c) = 0$, we must have $g_k = g_l$ for all $k,l = 0, \ldots, d+1$. In other words, $g_0 = g_1 = \cdots = g_{d+1}$. Consequently, our original $d$-chain can be rewritten as:
	\begin{equation} c = \sum_{i=0}^{n+1}g_0(v_0,\ldots,v_{i-1},v_{i+1},\ldots,v_d), \end{equation}
	allowing us to choose $g_0$ from $\mathbb{Z}$. Thus, we conclude that $H^\Delta_d(S^d) \cong \mathbb{Z}$.
	\item We demonstrate that $H^\Delta_d(D^d) = 0$. To do so, we employ the simplest simplicial structure for $D^d$, which is that of the $d$-simplex $\sigma^{(d)}$. Consequently, all $d$-chains can be expressed as $c = g \sigma^{(d)}$, where $g \in \mathbb{Z}$. It is important to note that this form is never a boundary, thus implying that $H^\Delta_d = Z^\Delta_d$. However, $\partial_d(c) = 0$ only when $g = 0$. Consequently, we can conclude that $H^\Delta_d(D^d) \cong 0$.
\end{itemize}
\end{example}

\section{Singular Homology}
In the realm of lower dimensions, we possess an intuitive understanding of when two topological spaces are fundamentally \glqq equivalent\grqq{}. To formalize and solidify this intuition, we have devised various methods, one of which is the concept of homeomorphism. It would be highly desirable to establish a relationship between the homology groups of homeomorphic spaces. Remarkably, it has been discovered that if two topological spaces are homeomorphic, their homology groups are isomorphic. This fact begs for verification.

To accomplish this task, we require a means of comparing homology groups. However, it is not immediately evident how we can achieve this using the tools we have developed thus far. In fact, it proves to be quite a challenging problem. To circumvent this difficulty, we introduce the notion of \emph{singular homology}. The fundamental principles underlying this concept are analogous to those we have already explored.
To compute $Z_n(X)$, we need to find the group of $n$-cycles in $X$. Since $X$ is obtained by identifying opposite faces of $\partial \sigma^{(n)}$, an $n$-cycle in $X$ corresponds to an $n$-cycle in $\partial \sigma^{(n)}$ that is not a boundary of any $(n+1)$-dimensional simplex in $\sigma^{(n)}$. In other words, an $n$-cycle in $X$ corresponds to an $n$-cycle in $\partial \sigma^{(n)}$ that is not a boundary of any $(n+1)$-dimensional face of $\sigma^{(n)}$.
\begin{definition}
In the context of a topological space $X$, we define a \textbf{singular $n$-simplex} as a map $\tilde{\sigma}^{(n)}: \sigma^{(n)} \rightarrow X$, where $\tilde{\sigma}^{(n)}$ is continuous.
\end{definition}

We define the boundary map $\partial_n$ in a similar manner as before:

\begin{definition}
The boundary map, denoted as $\partial_n$, is a function that operates on the chain group $C_n(X)$ and maps it to the chain group $C_{n-1}(X)$. It is defined as follows: For any singular $n$-simplex $\tilde{\sigma}^{(n)}$ in $X$, the boundary map $\partial_n(\tilde{\sigma}^{(n)})$ is obtained by summing over all the $(n-1)$-simplices that are obtained by removing one vertex from $\tilde{\sigma}^{(n)}$. Each term in the sum is multiplied by $(-1)^i$, where $i$ represents the index of the removed vertex. In other words, if $v_i$ represents the $0$-simplex (vertex) of $\tilde{\sigma}^{(n)}$, then the boundary map can be expressed as:
\begin{equation}
\partial_n(\tilde{\sigma}^{(n)}) = \sum_{i} (-1)^i \tilde{\sigma}^{(n)}\vert_{[v_0,\ldots,v_{i-1},v_{i+1},\ldots,v_n]}.
\end{equation}
Here, $v_i$ is a map that takes the $0$-simplex $\sigma^{(0)}$ to the corresponding vertex in $X$, s.t. $v_i: \sigma^{(0)} \rightarrow X$ is continuous.
\end{definition}

As mentioned earlier, when we apply the boundary map twice to an $n$-chain $c$, denoted as $\partial^2(c)$ or $\partial(\partial(c))$, the result is always zero. This observation leads us to the idea of defining the singular homology groups in a similar way to the simplicial homology groups.

\begin{definition}
The \textbf{singular homology group} $H_n(X)$ is defined to be the quotient $H_n(X) = \ker(\partial_n) / \text{im}(\partial_{n+1})$.
\end{definition}

In the following section, we will explore how this definition of homology allows us to establish a simple relationship between homeomorphic spaces and their corresponding homology groups. This relationship becomes apparent when we consider the fact that the definitions of $H_d$ and $H^\Delta_d$ are analogous. Intuitively, we would expect these two groups to be the same. However, this is not immediately obvious. One reason for this is that $H^\Delta_d$ is finitely generated, while the chain group $C_d(X)$, from which we derived $H_d$, is uncountable.

Interestingly, for spaces where both simplicial and singular homology groups can be calculated, these two groups are indeed equivalent. We will provide a proof for this later on. But before we do, let us present some facts about singular homology that support the intuition that $H_d$ is isomorphic to $H^\Delta_d$.

\begin{proposition}
In the context of a topological space $X$, it can be observed that $H_d(X)$ is isomorphic to the direct sum $H_d(X_1) \oplus \cdots \oplus H_d(X_p)$, where $X_i$ represents the path-connected components of $X$. This equivalence serves as the counterpart to Theorem \ref{decomptheorem}.\end{proposition}

\begin{proof}
As the maps $\tilde{\sigma}^{(d)}$ exhibit continuity, it can be deduced that a singular simplex always possesses a path-connected image within $X$. Consequently, $C_d(X)$ can be expressed as the direct sum of subgroups $C_d(X_1) \oplus \cdots \oplus C_d(X_p)$. The boundary map $\partial$ functions as a homomorphism, thereby preserving this decomposition. Consequently, $\ker(\partial_d)$ and $\text{im}(\partial_{d+1})$ also undergo a split, leading to the conclusion that $H_d(X) \cong H_d(X_1) \oplus H_d(X_2) \oplus \cdots \oplus H_d(X_p)$.
\end{proof}


\begin{proposition}
The zero-dimensional homology group of a space $X$ can be expressed as the direct sum of $\mathbb{Z}$ copies, with each copy corresponding to a distinct path-component of $X$. This correspondence serves as the parallel to Corollary \ref{directsum0hom}.\end{proposition}

\begin{proof}
To establish the isomorphism $H_0(X) \cong \mathbb{Z}$, it is sufficient to consider the case where $X$ is path-connected. For a $0$-chain $c$, the boundary operator $\partial_0(c)$ is always zero since the boundary of any $0$-simplex vanishes. Consequently, $\ker(\partial_0) = C_0(X)$, which implies that $H_0(X) = C_0(X) / \text{im}(\partial_1)$ by definition.

Let us define the map $I: C_0(X) \rightarrow \mathbb{Z}$, where $I(c) = \sum_i g_i$ for $c = \sum_i g_i \tilde{\sigma}^{(0)} \in C_0(X)$. Our goal is to demonstrate that $\ker(I) = \text{im}(\partial_1)$, or in other words, for any $0$-chain $c$, $I(c) = 0$ if and only if $c \sim 0$. The proof follows a similar line of reasoning as Proposition \ref{decomposition}.
\end{proof}

\begin{example}
Alternative proof that $H_d(S^d) \cong \mathbb{Z}$. To prove that the $d$th homology group of the $d$-sphere is isomorphic to $\mathbb{Z}$, we will use the singular homology approach. The $d$th singular chain group $C_d(S^d)$ consists of formal linear combinations of singular $d$-simplices in $S^d$ with integer coefficients. First, we note that $S^d$ is a connected and compact topological space. Therefore, by the Hurewicz theorem, we have $H_d(S^d) \cong \pi_d(S^d)$, where $\pi_d(S^d)$ denotes the $d$th homotopy group of $S^d$. Since $S^d$ is simply connected for $d \geq 2$, we have $\pi_d(S^d) = 0$ for $d \geq 2$. However, for $d = 1$, we have $\pi_1(S^1) \cong \mathbb{Z}$.

Now, we need to establish the isomorphism between $\pi_1(S^1)$ and $H_1(S^1)$. To do this, we consider the singular $1$-chain group $C_1(S^1)$, which consists of formal linear combinations of singular $1$-simplices in $S^1$ with integer coefficients. Let $c$ be a singular $1$-chain in $C_1(S^1)$. We can write $c$ as $c = \sum_{i} g_i \tilde{\sigma}^{(1)}_i$, where $g_i \in \mathbb{Z}$ and $\tilde{\sigma}^{(1)}_i$ are singular $1$-simplices. The boundary of $c$ is given by $\partial_1(c) = \sum_{i} g_i \partial(\tilde{\sigma}^{(1)}_i)$. Since $S^1$ is a $1$-dimensional manifold, the boundary of any singular $1$-simplex $\tilde{\sigma}^{(1)}_i$ is a formal linear combination of two points in $S^1$, each with opposite orientations. Therefore, $\partial(\tilde{\sigma}^{(1)}_i) = p - q$, where $p$ and $q$ are points in $S^1$. Hence, we have $\partial(c) = \sum_{i} g_i (p - q) = (p - q) \sum_{i} g_i$. Since $p$ and $q$ are fixed points in $S^1$, the sum $\sum_{i} g_i$ is an integer. Therefore, the boundary of any singular $1$-chain $c$ in $C_1(S^1)$ is of the form $(p - q)k$, where $k$ is an integer. This implies that $H_1(S^1) = Z_1(S^1) / B_1(S^1) \cong \mathbb{Z}$, where $Z_1(S^1)$ is the group of $1$-cycles and $B_1(S^1)$ is the group of $1$-boundaries. In conclusion, we have shown that $H_d(S^d) \cong \pi_d(S^d) = 0$ for $d \geq 2$, and $H_1(S^1) \cong \pi_1(S^1) \cong \mathbb{Z}$. Therefore, the $d$th homology group of the $d$-sphere is isomorphic to $\mathbb{Z}$. $\qed$
\end{example}

\section{Chain Complexes}
In order to establish the equivalence between the groups $H_d^\Delta$ and $H_d$, we will introduce some concepts that will aid us in our proof.

Firstly, since the maps $\tilde{\sigma}^{(d)}$ are continuous, it follows that a singular simplex always has a path-connected image in $X$. As a result, we can express $C_d(X)$ as the direct sum of subgroups $C_d(X_1) \oplus \cdots \oplus C_d(X_p)$, where each subgroup corresponds to a distinct path-component of $X$. This decomposition is preserved by the boundary map $\partial$, which is a homomorphism. Consequently, both $\ker(\partial_d)$ and $\text{im}(\partial_{d+1})$ also split, leading to the conclusion that $H_d(X)$ is isomorphic to $H_d(X_1) \oplus H_d(X_2) \oplus \cdots \oplus H_d(X_p)$.

These ideas will serve as valuable tools in our endeavor to prove the equivalence of the groups $H_d^\Delta$ and $H_d$.

\begin{definition}
A \textbf{chain complex} is an arrangement of abelian groups, linked together by homomorphisms (referred to as boundary operators), in such a way that the result of combining any two consecutive maps is precisely zero.
\end{definition}

\begin{example}
The groups $C_d(X)$ represent the collection of singular $d$-chains that form a part of a chain complex, where the boundary operator $\partial_d$ guides the flow between these groups.
\begin{equation}
\cdots \xrightarrow{} C_{d+1} \xrightarrow{\partial_{d+1}} C_d \xrightarrow{\partial_d} C_{d-1} \xrightarrow{} \cdots \xrightarrow{} C_1 \xrightarrow{\partial_1} C_0 \xrightarrow{\partial_0} 0.
\end{equation}
\end{example}

\begin{definition}
A \textbf{chain map} $f$ between two chain complexes $(A, \partial_A)$ and $(B,\partial_B)$ is a collection of maps $f_d: A_d \rightarrow B_d$ such that for each $d$, the following conditions hold:
\begin{itemize}
    \item $f_d$ commutes with the operator $\partial_A$, i.e., $\partial_A \circ f_d = f_{d-1} \circ \partial_A$.
    \item $f_d$ commutes with the operator $\partial_B$, i.e., $\partial_B \circ f_d = f_{d-1} \circ \partial_B$.
\end{itemize}
\end{definition}

\begin{equation}
\begin{tikzcd}
\cdots \arrow[r, "\partial_A"] & A_{d+1} \arrow[r, "\partial_A"] \arrow[d, "f_{d+1}"] & A_d \arrow[r, "\partial_A"] \arrow[d, "f_d"] & A_{d-1} \arrow[r, "\partial_A"] \arrow[d, "f_{d-1}"] & \cdots \\
\cdots \arrow[r, "\partial_B"] & B_{d+1} \arrow[r, "\partial_B"]                                  & B_d \arrow[r, "\partial_B"]                              & B_{d-1} \arrow[r, "\partial_B"]                                  & \cdots
\end{tikzcd}
\end{equation}

\begin{theorem}
\label{chainmaps}
A chain map $f$ between two chain complexes $(A, \partial_A)$ and $(B,\partial_B)$ induces a homomorphism between their respective homology groups.
\end{theorem}

\begin{proof}
Given a chain map $f$ between two chain complexes $(A, \partial_A)$ and $(B,\partial_B)$, we want to show that $f$ induces a homomorphism $f_\star: H_d(A) \rightarrow H_d(B)$. By the definition of a chain map, we have $f \circ \partial_A = \partial_B \circ f$. Let $[c] \in Z_d(A)$ be a cycle in $A$, i.e., $\partial_A(c) = 0$. Applying $f$ to both sides of this equation, we get $f(\partial_A(c)) = \partial_B(f(c))$. Since $\partial_A(c) = 0$, we have $f(0) = \partial_B(f(c))$, which implies $\partial_B(f(c)) = 0$. Therefore, $f(c)$ is a cycle in $B$. Now, let $[b] \in B_d(A)$ be a boundary in $A$, i.e., there exists $a \in A_{d+1}$ such that $\partial_A(a) = b$. Applying $f$ to both sides of this equation, we get $f(\partial_A(a)) = \partial_B(f(a))$. Since $\partial_A(a) = b$, we have $f(b) = \partial_B(f(a))$. Therefore, $f(b)$ is a boundary in $B$. From the above, we see that $f$ maps cycles in $A$ to cycles in $B$ and boundaries in $A$ to boundaries in $B$. Hence, $f$ induces a well-defined map $f_\star: H_d(A) \rightarrow H_d(B)$, where $H_d(A)$ and $H_d(B)$ are the homology groups of $A$ and $B$ at dimension $d$, respectively. To show that $f_\star$ is a homomorphism, let $[c_1], [c_2] \in H_d(A)$ be two homology classes. We want to show that $f_\star([c_1] + [c_2]) = f_\star([c_1]) + f_\star([c_2])$. Let $c_1$ and $c_2$ be representatives of $[c_1]$ and $[c_2]$ respectively. Then, $[c_1] + [c_2]$ is represented by $c_1 + c_2$. Applying $f$ to both sides, we have $f(c_1 + c_2) = f(c_1) + f(c_2)$. Since $f(c_1)$ and $f(c_2)$ are cycles in $B$, we have $[f(c_1 + c_2)] = [f(c_1)] + [f(c_2)]$. Therefore, $f_\star([c_1] + [c_2]) = f_\star([c_1]) + f_\star([c_2])$. Hence, we have shown that the chain map $f$ induces a homomorphism $f_\star: H_d(A) \rightarrow H_d(B)$.
\end{proof}

\section{Exact and Short Exact Sequences}
We can apply Theorem \ref{chainmaps} to the case of singular homology. Consider two topological spaces $X$ and $Y$. For any map $f: X \rightarrow Y$, we can define an induced homomorphism $f_\star: C_d(X) \rightarrow C_d(Y)$ by composing singular $d$-simplices $\tilde{\sigma}^{(d)}: \sigma^{(d)} \rightarrow X$ with $f$. Specifically, we have $f_\star \circ \tilde{\sigma}^{(d)} = f \circ \tilde{\sigma}^{(d)}: \sigma^{(d)} \rightarrow Y$.

We can extend this definition by applying $f_\star$ to $d$-chains in $C_d(X)$. This gives us a commutative diagram.

\begin{equation}
\begin{tikzcd}
\cdots \arrow[r, "\partial"] & C_{d+1}(X) \arrow[r, "\partial"] \arrow[d, "f_{d+1}"] & C_d(X) \arrow[r, "\partial"] \arrow[d, "f_d"] & C_{d-1}(X) \arrow[r, "\partial"] \arrow[d, "f_{d-1}"] & \cdots \\
\cdots \arrow[r, "\partial"] & C_{d+1}(Y) \arrow[r, "\partial"]                                  & C_d(Y) \arrow[r, "\partial"]                              & C_{d-1}(Y) \arrow[r, "\partial"]                                  & \cdots
\end{tikzcd}
\end{equation}

The chain map $f_d$ gives rise to a homomorphism $f_\star: H_d(X) \rightarrow H_d(Y)$. It becomes evident that if $X$ and $Y$ are homeomorphic, meaning there exists a homeomorphism $f: X \rightarrow Y$, then the induced map $f_\star$ is an isomorphism.

To formalize the relationships between the homology groups of a topological space $X$, a subset $A \subset X$, and the quotient space $X/A$, we introduce the concept of \emph{exact sequences}.

\begin{definition}
An arrangement of elements in the form \begin{equation}
\cdots \rightarrow A_{d+1} \xrightarrow{\alpha_{n+1}} A_{d} \xrightarrow{\alpha_d} A_{d-1} \xrightarrow{} \cdots
\end{equation}
is referred to as an \textbf{exact sequence} when the $A_i$ are abelian groups and the $\alpha_i$ are homomorphisms, and it satisfies the condition that $\ker(\alpha_d) = \text{im}(\alpha_{d+1})$ for all $d$.\end{definition}

\paragraph{Note:}
\begin{itemize}
	\item The condition $\ker(\alpha_d) = \text{im}(\alpha_{d+1})$ implies that $\text{im}(\alpha_{d+1})$ is a subset of $\ker(\alpha_d)$, which is equivalent to $\alpha_d \circ \alpha_{d+1} = 0$. Therefore, an exact sequence can be seen as a chain complex.
	\item Since $\ker(\alpha_d)$ is a subset of $\text{im}(\alpha_{d+1})$, the homology groups of an exact sequence are trivial.
\end{itemize}

\begin{proposition}
We can establish the following euqivalences:
\begin{enumerate}
	\item $0 \xrightarrow{} A \xrightarrow{a} B$ is exact $\Longleftrightarrow$ $\ker(a) = 0$, or $a$ is injective.
	\item $A \xrightarrow{a} B \rightarrow 0$ is exact $\Longleftrightarrow$ $\text{im}(a) = B$, or $a$ is surjective.
	\item $0 \xrightarrow{} A \xrightarrow{a} B \xrightarrow 0$ is exact if and only if $a$ is an isomorphism.
	\item A sequence of the form $0 \xrightarrow{} A \xrightarrow{a} B \xrightarrow{} 0$ is said to be exact if and only if the following conditions hold:
	\begin{itemize}
		\item The map $a: A \rightarrow B$ is injective, meaning that $\ker(a) = 0$.
		\item The map $b: B \rightarrow 0$ is surjective, meaning that $\text{im}(b) = 0$.
		\item The kernel of $b$ is equal to the image of $a$, i.e., $\ker(b) = \text{im}(a)$.
		\item In this case, the map $b$ induces an isomorphism $C \cong B/\text{im}(a)$, where $C$ is the quotient of $B$ by the image of $a$.
	\end{itemize}
	If the map $a: A \hookrightarrow B$ is an inclusion, then $C \cong B/A$, where $B/A$ denotes the quotient of $B$ by the subgroup $A$. This type of exact sequence is commonly referred to as a \textbf{short exact sequence}.
\end{enumerate}
\end{proposition}

\begin{proof}
\begin{enumerate}
	\item[]
	\item[1.] \glqq $\Rightarrow$\grqq{}: First, assume that $0 \xrightarrow{} A \xrightarrow{a} B$ is exact. This means that $\text{im}(0) = \ker(a)$, which implies that $\ker(a) = 0$ since the image of the zero map is always the trivial group. Therefore, $\ker(a) = 0$.

	\glqq $\Leftarrow$\grqq{}: Conversely, suppose that $\ker(a) = 0$ or $a$ is injective. We need to show that $\text{im}(0) = \ker(a)$. Since $\ker(a) = 0$, the only element mapped to the identity element in $B$ is the zero element of $A$. This means that the image of the zero map is equal to the kernel of $a$, which implies that $0 \xrightarrow{} A \xrightarrow{a} B$ is exact.
	\item[2.] \glqq $\Rightarrow$\grqq{}: First, assume that $A \xrightarrow{a} B \xrightarrow{} 0$ is exact. This means that $\text{im}(a) = \ker(0)$, which implies that $\text{im}(a) = B$ since the kernel of the zero map is always the entire group. Therefore, $\text{im}(a) = B$.

	\glqq $\Leftarrow$\grqq{}: Conversely, suppose that $\text{im}(a) = B$ or $a$ is surjective. We need to show that $\text{im}(a) = \ker(0)$. Since $\text{im}(a) = B$, every element in $B$ has a preimage in $A$ under the map $a$. This means that the image of $a$ is equal to the kernel of the zero map, which implies that $A \xrightarrow{a} B \rightarrow 0$ is exact.
	\item[3.] \glqq $\Rightarrow$\grqq{}: First, assume that $0 \xrightarrow{} A \xrightarrow{a} B \xrightarrow{} 0$ is exact. This means that $\text{im}(0) = \ker(a)$ and $\text{im}(a) = \ker(0)$. Since the image of the zero map is always the trivial group, we have $\ker(0) = B$. Therefore, $\text{im}(a) = B$. Since $\text{im}(a) = B$, this implies that $a$ is surjective. To show that $a$ is also injective, we need to show that $\ker(a) = 0$. From the exactness of the sequence, we have $\ker(a) = \text{im}(0) = 0$. Therefore, $a$ is injective. Since $a$ is both surjective and injective, it is an isomorphism.

	\glqq $\Leftarrow$\grqq{}: Conversely, suppose that $a$ is an isomorphism. This means that $a$ is both surjective and injective. Since $a$ is surjective, we have $\text{im}(a) = B$. Since $a$ is injective, we have $\ker(a) = 0$. From the exactness of the sequence, we have $\text{im}(a) = \ker(0)$. Since the kernel of the zero map is always the entire group, we have $\ker(0) = B$. Therefore, $\text{im}(a) = B$.
	\item \glqq $\Rightarrow$\grqq{}: First, assume that the sequence is exact. This means that $a$ is injective, $\text{im}(a) = \ker(b)$, and $b$ is surjective. Since $b$ is a map from $B$ to $0$, the only possible value for $\text{im}(b)$ is the zero element. Therefore, $\text{im}(b) = 0$. Since $\text{im}(a) = \ker(b)$, we have $\ker(b) = 0$. This implies that $b$ is the zero map, and since $b$ is surjective, the only possible value for $B$ is the zero element. Therefore, $B = 0$. Since $a$ is injective and $B = 0$, the only possible value for $A$ is also the zero element of $0$. Therefore, $A = 0$. We have shown that $A = 0$, $B = 0$, and $a$ is injective. This implies that $a$ is an isomorphism.

	\glqq $\Leftarrow$\grqq{}: Conversely, suppose that $a$ is an isomorphism. This means that $a$ is injective and surjective. Since $a$ is injective, $\ker(a) = 0$. Since $a$ is surjective, $\text{im}(a) = B$. Therefore, $\ker(a) = 0 = \text{im}(a)$. Since $\text{im}(a) = \ker(b)$, we have $\ker(b) = 0 = \text{im}(a)$. This implies that $b$ is the zero map. Since $b$ is surjective, the only possible value for $B$ is the zero element. Therefore, $B = 0$. Since $B = 0$, the only possible value for $\text{im}(b)$ is also the zero element. Therefore, $\text{im}(b) = 0$. We have shown that $B = 0$, $\text{im}(b) = 0$, and $\ker(b) = \text{im}(a)$. This implies that the sequence is exact.
\end{enumerate}
\end{proof}

\section{Relative Homology Groups}
The concept we will now discuss is that of \emph{relative homology groups}. Let $X$ be a topological space and $A$ be a subspace of $X$. We define $C_d(X,A)$ as the quotient group $C_d(X)/C_d(A)$. This means that chains in $A$ are considered equivalent to the trivial chains in $C_d(X)$.

Since the boundary operator $\partial: C_d(X) \rightarrow C_{d-1}(X)$ also maps $C_d(A)$ to $C_{d-1}(A)$, we obtain a natural boundary map on the quotient group $\partial: C_d(X,A) \rightarrow C_{d-1}(X,A)$. This gives rise to the following sequence:

\begin{equation}
\cdots \xrightarrow{} C_{d+1}(X,A) \xrightarrow{\partial_{d+1}} C_d(X,A) \xrightarrow{\partial_d} C_{d-1}(X,A) \rightarrow \cdots
\end{equation}

This sequence forms a chain complex because $\partial_{d+1}\circ\partial_{d} = 0$. We can then define the \emph{relative homology groups} $H_d(X,A)$ as the homology groups of this chain complex.

We propose two important facts about $H_d(X,A)$:

\begin{proposition}
\begin{enumerate}
	\item[]
	\item[1.] Elements in $H_d(X,A)$ are represented by \textbf{relative cycles}, or $d$-chains $c$ in $C_d(X)$ such that $\partial_d(c) = C_{d-1}(A)$.
	\item[2.] A relative cycle $c$ is trivial if and only if it is a \textbf{relative boundary}, i.e. $c$ is the sum of a chain in $C_d(A)$ and the boundary of a chain in $C_{d+1}(X)$.
\end{enumerate}
\end{proposition}

\begin{proof}
\begin{enumerate}
	\item[]
	\item[1.] \glqq $\Rightarrow$\grqq{}: Let us begin by considering an element $[c] \in H_d(X,A)$, where $[c]$ represents the homology class of $c$ in $H_d(X,A)$. By definition, $[c]$ is the set of all $d$-chains $c'$ in $C_d(X)$ that are homologous to $c$, denoted by $\partial_d(c') = \partial_d(c)$. Now, let's examine the boundary of $c$ under the boundary operator $\partial_d$. We observe that $\partial_d(c) \in C_{d-1}(X)$, indicating that $\partial_d(c)$ is a $(d-1)$-chain in $X$. Since $A$ is a subspace of $X$, we can consider the inclusion map $f: A \hookrightarrow X$. This map induces a homomorphism $f_\star: C_{d-1}(A) \rightarrow C_{d-1}(X)$ between the chain groups. Next, let's consider the image of $\partial_d(c)$ under the map $i$. We have $f(\partial_d(c)) \in C_{d-1}(X)$, implying that $f(\partial_d(c))$ is a $(d-1)$-chain in $X$. As $f$ is a homomorphism, we have $f(\partial_d(c)) = \partial_d(f(c))$, where $f_\star(c)$ is a $(d-1)$-chain in $A$. Consequently, we can conclude that $\partial_d(c) = \partial_d(f(c))$, where $\partial_d(c) \in C_{d-1}(X)$ and $\partial_d(f(c)) \in C_{d-1}(A)$. This demonstrates that $c$ is a relative cycle, as it satisfies the condition $\partial_d(c) \in C_{d-1}(A)$.

	\glqq $\Leftarrow$\grqq{}: Conversely, if we possess a relative cycle $c$ in $C_d(X)$ such that $\partial_d(c) \in C_{d-1}(A)$, we can consider the homology class $[c]$ in $H_d(X,A)$. This class represents the set of all $d$-chains in $C_d(X)$ that are homologous to $c$, denoted by $\partial_d(c') = \partial_d(c)$. Hence, we have successfully demonstrated that elements in $H_d(X,A)$ are represented by relative cycles, which are $d$-chains $c$ in $C_d(X)$ such that $\partial_d(c) \in C_{d-1}(A)$.
	\item[2.] \glqq $\Rightarrow$\grqq{}: First, let's assume that $c$ is a relative boundary. This implies that $c$ can be written as the sum of a chain in $C_d(A)$, denoted as $a$, and the boundary of a chain in $C_{d+1}(X)$, denoted as $b$. Therefore, we have $c = a + \partial_{d+1}(b)$. Now, let's consider the boundary of $c$. We have $\partial_d(c) = \partial_d(a + \partial_{d+1}(b))$. By utilizing the linearity of the boundary operator, we can rewrite this as $\partial_d(a) + \partial_d(\partial_{d+1}(b))$. Since $a$ is a chain in $C_d(A)$, its boundary $\partial_d(a)$ lies in $C_{d-1}(A)$. Additionally, we know that $\partial_d(\partial_{d+1}(b)) = 0$ because the boundary of a boundary is always zero. Therefore, we have $\partial_d(c) = \partial_d(a) + \partial_d(\partial_{d+1}(b)) = \partial_d(a) + 0 = \partial_d(a)$. As $\partial_d(a)$ is in $C_{d-1}(A)$, we can conclude that $\partial_d(c)$ is also in $C_{d-1}(A)$. This demonstrates that $c$ is a relative cycle.

	\glqq $\Leftarrow$\grqq{}: Conversely, let's assume that $c$ is a trivial relative cycle, indicating that $\partial_d(c)$ is in $C_{d-1}(A)$. Since $\partial_d(c)$ lies in $C_{d-1}(A)$, we can express it as the boundary of a chain in $C_d(A)$. Let's denote this chain as $a$. Therefore, we have $\partial_d(c) = \partial_d(a)$. Now, let's consider the chain $b = c - a$. We have $\partial_d(b) = \partial_d(c - a) = \partial_d(c) - \partial_d(a) = \partial_d(c) - \partial_d(c) = 0$. This demonstrates that $b$ is a chain in $C_d(X)$ whose boundary is zero. Hence, we have successfully shown that $c$ can be expressed as the sum of a chain in $C_d(A)$ (represented by $a$) and the boundary of a chain in $C_{d+1}(X)$ (represented by $b$). Consequently, $c$ is a relative boundary.

	In conclusion, we have proven that a relative cycle $c$ is trivial if and only if it is a relative boundary, meaning that $c$ can be expressed as the sum of a chain in $C_d(A)$ and the boundary of a chain in $C_{d+1}(X)$.
\end{enumerate}
\end{proof}

\begin{theorem}
The relative homology groups $H_d(X,A)$ are part of the exact sequence:
\begin{equation*}
\cdots \rightarrow H_d(A) \rightarrow H_d(X) \rightarrow H_d(X,A) \rightarrow H_{d-1}(A) \rightarrow \cdots \rightarrow H_0(X,A) \rightarrow 0.
\end{equation*}
\end{theorem}

\begin{proof}
Consider the following diagram:
\begin{equation}
\begin{tikzcd}
0 \arrow[r] & C_d(A) \arrow[r, "i", hook] \arrow[d, "\partial"] & C_d(X) \arrow[r, "j", two heads] \arrow[d, "\partial"] & {C_d(X,A)} \arrow[r] \arrow[d, "\partial"] & 0 \\
0 \arrow[r] & C_{d-1}(A) \arrow[r, "i", hook]                               & C_{d-1}(X) \arrow[r, "j", two heads]                               & {C_{d-1}(X,A)} \arrow[r]                               & 0.
\end{tikzcd}
\end{equation}
$i$ is the inclusion $C_d(A) \hookrightarrow C_d(X)$ and $j$ is the quotient map $C_d(X) \twoheadrightarrow C_d(X,A)$. This diagram is commuting, and turning it by $90$ degrees yields the following diagram for abelian groups $A_i, B_i$ and $C_i$:
\begin{equation}
\begin{tikzcd}
                 & 0 \arrow[d]                                                         & 0 \arrow[d]                                                     & 0 \arrow[d]                                                         &        \\
\cdots \arrow[r] & A_{d+1} \arrow[r, "\partial"] \arrow[d, "i", hook]      & A_d \arrow[r, "\partial"] \arrow[d, "i", hook]      & A_{n-1} \arrow[r, "\partial"] \arrow[d, "i", hook]      & \cdots \\
\cdots \arrow[r] & B_{d+1} \arrow[r, "\partial"] \arrow[d, "j", two heads] & B_d \arrow[r, "\partial"] \arrow[d, "j", two heads] & B_{d-1} \arrow[r, "\partial"] \arrow[d, "j", two heads] & \cdots \\
\cdots \arrow[r] & C_{d+1} \arrow[r, "\partial"] \arrow[d]                             & C_d \arrow[r, "\partial"] \arrow[d]                             & C_{d-1} \arrow[r, "\partial"] \arrow[d]                             & \cdots \\
                 & 0                                                                   & 0                                                               & 0                                                                   &       .
\end{tikzcd}
\end{equation}
Presenting the diagram in this manner indicates that $i$ and $j$ are chain maps, and therefore induce maps $i_\star$ and $j_\star$ on homology, as stated in Theorem \ref{chainmaps}. Let us consider a cycle $c \in C_d$. Since $j$ is surjective, there exists a chain $b \in B_d$ such that $c = j(b)$. By the commutativity of the diagram, we have $j(\partial(b)) = \partial(j(b))$. As $c$ is a cycle, we have $\partial(c) = \partial(j(b)) = 0$. This implies that $\partial(b) \in \ker(j)$. Since the columns of the diagram are exact, we know that $\ker(j) = \text{im}(i)$. Therefore, there exists an element $a \in A_{d-1}$ such that $\partial(b) = i(a)$. By the commutativity of the diagram, we have $i(\partial(a)) = \partial(i(a)) = \partial(\partial(b))$. This leads to the implication that if $i$ is injective, then $\partial(a) = 0$. Hence, $a$ is a cycle and represents an element $[a] \in H_{d-1}(A)$ in the homology. We can now define the map $\partial: H_d(C) \rightarrow H_{d-1}(A)$ by sending the homology class of $[c]$ to the homology class of $[a]$, denoted as $\partial([c]) = [a]$. This definition is well-defined due to:
\begin{enumerate}
	\item Since $i$ is injective, the value of $a$ is uniquely determined by $\partial(b)$.
	\item If we choose a different chain $b'$ instead of $b$, we have $j(b') = j(b) \implies j(b')-j(b) = 0 \implies j(b'-b) = 0 \implies b-b' \in \ker(j) = \text{im}(i)$. Therefore, $b-b' = i(a')$, or $b' = b+i(a')$. This implies that $\partial(b+i(a')) = \partial(b) + \partial(i(a')) = i(a) + i \partial(a') = i(a + \partial(a'))$. However, since $\partial(a') \sim 0$, we have $a+\partial(a') \sim a$.
	\item If we select $c_\star$ from the coset containing $c$, it means that $c_\star = c + \partial(c')$. Since $c'$ can be expressed as $j(b')$ for some $b'$, we can rewrite $c+\partial(c')$ as $c+\partial(j(b')) = j(b) + j\partial(b') = j(b+\partial(b'))$. Therefore, modifying $c$ results in a corresponding modification of $b$ to a homologous element, but this does not have any impact on $a$ whatsoever.
\end{enumerate}
\end{proof}

\begin{proposition}
The map $\partial: H_d(C) \rightarrow H_{d-1}(A)$ is a homomorphism.
\end{proposition}

\begin{proof}
If $\partial([c_1])$ yields $[a_1]$ and $\partial([c_2])$ yields $[a_2]$, as indicated earlier, then we can conclude that $j(b_1 + b_2) = j(b_1) + j(b_2) = c_1 + c_2$, and $i(a_1 + a_2) = \partial(b_1) + \partial(b_2)$. Consequently, it follows that $\partial([c_1] + [c_2]) = [a_1] + [a_2]$.
\end{proof}

\begin{lemma}
\label{exacthomsequence}
The given sequence,
\begin{equation*}
\cdots \rightarrow H_d(A) \xrightarrow{i_\star} H_d(B) \xrightarrow{j_\star} H_d(C) \xrightarrow{\partial} H_{d-1}(A) \xrightarrow{i_\star} H_{d-1}(B) \rightarrow \cdots
\end{equation*}
is said to be exact.
\end{lemma}

\begin{proof}
There are six inclusions that need to be confirmed:
\begin{enumerate}
	\item Inclusion of the image of $i_\star$ within the kernel of $j_\star$: Because $j\circ i = 0$, it implies that $j_\star \circ i_\star = 0$.
	\item Inclusion of the image of $j_\star$ within the kernel of $\partial$: By definition, $\partial(b) = 0$, so $\partial \circ j_\star = 0$.
	\item Inclusion of the image of $\partial$ within the kernel of $i_\star$: We have $i_\star \circ \partial = 0$ since $i_\star \circ \partial([c]) = [\partial(b)] = 0$.
	\item Inclusion of the kernel of $j_\star$ within the image of $i_\star$: A homology class in the kernel of $j_\star$ can be represented by a cycle $b \in B_d$ such that $j(b) = \partial(c')$ is a boundary for some $c' \in C_{d+1}$. The surjectivity of $j$ implies that $c' = j(b')$ for some $b' \in B_{d+1}$. Consequently, we have $j(b) = \partial(c') = \partial \circ j(b')$, which leads to $j(b - \partial(b')) = 0$. Therefore, $b - \partial(b') = i(a)$ for some $a \in A_d$. Since $i$ is injective, $a$ is a cycle because $i \circ \partial(a) = \partial \circ i(a) = \partial(b - \partial(b')) = \partial(b) = 0$, given that $b$ is a cycle. Consequently, $i_\star[a] = [b]$, and the two inclusions demonstrate that $\text{im}(i_\star) = \ker(j_\star)$.
	\item Inclusion of the kernel of $\partial$ within the image of $j_\star$: Let's consider a representative $c$ of a homology class in $\ker(\partial)$. This means we have $a = \partial(a')$ for some $a' \in A_d$. The element $b - i(a')$ is a cycle because $\partial(b - i(a')) = \partial(b) - \partial \circ i(a') = \partial(b) - i(\partial(a')) = \partial(b) - i(a) = 0$. Furthermore, we have $j(b - i(a')) = j(b) - j \circ i(a') = j(b) = c$. Thus, we conclude that $\ker(\partial) \subset \text{im}(j_\star)$.
	\item Inclusion of the kernel of $i_\star$ within the image of $\partial$: Consider a cycle $a$ in $A_{d-1}$ such that $i(a) = \partial(b)$ for some $b$ in $B_d$. Since $\partial(j(b)) = j(\partial(b)) = j \circ i(a) = 0$, we can determine that $j(b)$ is also a cycle. Consequently, $\partial([j(b)]) = [a]$, which demonstrates that $\ker(i_\star) \subset \text{im}(\partial)$.
\end{enumerate}
Indeed, we have established the following relationships:
\begin{itemize}
    \item $\text{im}(i_\star) = \ker(j_\star)$,
    \item $\text{im}(j_\star) = \ker(\partial)$,
    \item $\text{im}(\partial) = \ker(i_\star)$.
\end{itemize}
These relationships confirm that the sequence is exact, as it satisfies all the necessary conditions for exactness.\end{proof}

\begin{proposition}
The given sequence,
\begin{equation*}
\cdots \rightarrow H_d(A) \xrightarrow{i_\star} H_d(X) \xrightarrow{j_\star} H_d(X,A) \xrightarrow{\partial} H_{d-1}(A) \rightarrow \cdots \rightarrow H_d(X,A) \rightarrow 0
\end{equation*}
is said to be exact.
\end{proposition}

\begin{proof}
The conclusion follows from the previous Proposition \ref{exacthomsequence}, with the additional observation that for a relative cycle $c$ in $H_d(X,A)$, the application of $\partial([c])$ results in the class of the cycle $[\partial(c)]$ in $H_{d-1}(A)$.
\end{proof}

Additionally, we make reference to the Excision Theorem, a well-established result within this field. While the theorem's statement appears straightforward and intuitively sound, its actual proof can be rather intricate. Therefore, we present the theorem without delving into the proof details here.

In simpler terms, the Excision Theorem enables us to investigate the homology of a space by effectively \glqq cutting out\grqq{} a smaller subspace, provided certain conditions regarding the relationship between these subspaces are satisfied. In essence, this theorem assures us that, under the specified conditions, the homology of the original space remains unchanged when compared to the homology of the space resulting from the removal of the smaller subspace.

\begin{theorem}{(The Excision Theorem)}
\label{excisiontheorem}
Given $Y \subset A \subset X$, with the closure of $Y$ contained in the interior of $A$, then $(X\setminus Y, A \setminus Y) \hookrightarrow (X,A)$ induces isomorphisms $H_d(X\setminus Y, A\setminus Y) \rightarrow H_d(X,A)$ for all $d$.
\end{theorem}

\section{Equivalence of Simplicial Homology Group $H_d^\Delta$ and Singular Homology Group $H_d$}
We aim to establish the equivalence between the groups $H_d(X)$ and $H^\Delta_d(X)$. It is important to note that simplicial homology groups are defined and computable only for simplicial structures. However, this limitation can be overcome by calculating singular homology groups for any topological space, including simplicial complexes. Furthermore, the fact that homeomorphic spaces have isomorphic singular homology groups suggests that we can impose a simplicial structure on a topological space. Therefore, to prove the equivalence of $H_d(X)$ and $H^\Delta_d(X)$, we consider an arbitrary simplicial complex as our topological space $X$. It is worth mentioning that not all topological spaces can be homeomorphic to a simplicial complex, but for the purpose of this paper, we will focus solely on spaces that can be represented as simplicial complexes.

To demonstrate the equivalence of $H_d(X)$ and $H^\Delta_d(X)$, we need to establish the existence of an isomorphism between the two groups for all $d$. It is relatively straightforward to observe the existence of a homomorphism: we already have a map $L_d(X) \rightarrow C_d(X)$ from the simplicial chain group to the singular chain group, which maps each simplex of $X$ to $\tilde{\sigma}^{(d)}: \sigma^{(d)} \rightarrow X$. This map induces a corresponding map $H^\Delta_d(X) \rightarrow H_d(X)$.

\begin{lemma}{(The Five Lemma)}
\label{fivelemma}
In a commutative diagram structured as follows:
\begin{equation}
\begin{tikzcd}
A \arrow[r, "i"] \arrow[d, "\alpha"] & B \arrow[r, "j"] \arrow[d, "\beta"] & C \arrow[r, "k"] \arrow[d, "\gamma"] & D \arrow[r, "l"] \arrow[d, "\delta"] & E \arrow[d, "\epsilon"] \\
A' \arrow[r, "i'"]                               & B' \arrow[r, "j'"]                              & C' \arrow[r, "k'"]                               & D' \arrow[r, "l'"]                               & E'.                                 
\end{tikzcd}
\end{equation}
If the morphisms $\alpha, \beta, \delta, \epsilon$ are all isomorphisms, and both rows in the diagram are exact, then it follows that $\gamma$ is also an isomorphism.
\end{lemma}

\begin{proof}
The commutativity of the diagram implies that $\gamma$ is a homomorphism. To establish that $\gamma$ is bijective, we proceed as follows:
\begin{itemize}
    \item Surjectivity of $\gamma$: Let $c' \in C'$. Since $\delta$ is surjective, there exists some $d \in D$ such that $k'(c') = \delta(d)$. Injectivity of $\epsilon$ implies $\epsilon \circ l(d) = l' \circ \delta(d) = l' \circ k'(c') = 0$. Therefore, $l(d) = 0$. Since the rows are exact, we have $d = k(c)$ for some $c \in C$. This leads to $k'(c') - k'(\gamma(c)) = k'(c') - \delta \circ k(c) = k'(c') - \delta(d) = 0$. Hence, $k'(c' - \gamma(c)) = 0$, and by exactness, $c' - \gamma(c) = j'(b')$ for some $b' \in B'$. The surjectivity of $\beta$ implies that $b' = \beta(b)$ for some $b \in B$. Consequently, $\gamma(c + j(b)) = \gamma(c) + \gamma(j(b)) = \gamma(c) + j' \circ \beta(b) = \gamma(c) - j'(b') = c'$, establishing the surjectivity of $\gamma$.
    \item Injectivity of $\gamma$: Suppose $\gamma(c) = 0$. Since $\delta$ is injective, we have $\delta(k(c)) = k'(\gamma(c)) = 0$. This implies $k(c) = 0$. Therefore, $c = j(b)$ for some $b \in B$. From $\gamma(c) = \gamma(j(b)) = j'(\beta(b))$, we deduce that $\beta(b) = i'(a')$ for some $a' \in A'$. Surjectivity of $\alpha$ gives us $a' = \alpha(a)$ for some $a \in A$. As $\beta$ is injective, we find that $\beta(i(a) - b) = \beta(i(a)) - \beta(b) = i'(\alpha(a)) - \beta(b) = i'(a') - \beta(b) = 0$. Therefore, $i(a) - b = 0$, which implies $b = i(a)$. Consequently, $c = j(b) = j(i(a)) = 0$ by the exactness of rows. Hence, $\gamma$ has a trivial kernel and is thus injective.
\end{itemize}
In summary, we have shown that $\gamma$ is both surjective and injective, which establishes its bijectiveness.
\end{proof}

Simplicial and singular homology, although stemming from similar concepts, have distinct purposes in algebraic topology. They excel in different scenarios: Simplicial homology simplifies homology group calculations, particularly suited for geometric problems, while singular homology is advantageous when streamlined theorem proofs are needed, thanks to its compatibility with continuous maps.

The fundamental achievement in algebraic topology lies in their equivalence. This unification not only bridges the gap between the two approaches but also equips mathematicians with a versatile tool for addressing a wide range of problems. It stands as a pivotal result in the field, empowering researchers to tackle diverse challenges effectively.

\begin{theorem}{(Equivalence of Simplicial and Singular Homology)}
For all values of $d$, the homomorphisms from the simplicial homology group $H^\Delta_d(X)$ to the singular homology group $H_d(X)$ are isomorphisms. Therefore, it follows that the singular and simplicial homology groups are equivalent.
\end{theorem}

\begin{proof}
Consider a simplicial complex $X$. For the $k$-skeleton $X^k$ of $X$, we obtain the following commutative diagram of exact sequences due to the inclusion $X^{k-1} \subset X^k$:
\begin{equation*}
\begin{tikzcd}[column sep=1.2em]
{H^\Delta_{d+1}(X^k,X^{k-1})} \arrow[d] \arrow[r] & H^\Delta_{d}(X^{k-1}) \arrow[d] \arrow[r] & H^\Delta_{d}(X^k) \arrow[d] \arrow[r] & {H^\Delta_{d}(X^k,X^{k-1})} \arrow[d] \arrow[r] & H^\Delta_{d-1}(X^{k-1}) \arrow[d] \\
{H_{d+1}(X^k,X^{k-1})} \arrow[r]                  & H_{d}(X^{k-1}) \arrow[r]                  & H_{d}(X^k) \arrow[r]                  & {H_{d}(X^k,X^{k-1})} \arrow[r]                  & H_{d-1}(X^{k-1})                 
\end{tikzcd}
\end{equation*}
Here, $X^k/X^{k-1}$ contains only simplices of dimension $k$, so for $d \neq k$, the group $L_d(X^k, X^{k-1})$ is trivial. When $d = k$, $L_d(X^k, X^{k-1})$ is a free abelian group with a basis consisting of the $k$-simplices of $X$. Since the cycles $Z_d$ form a subgroup of $L_d$, and the boundary group $B_d$ is empty, $H^\Delta_d(X^k, X^{k-1})$ is essentially the same as $L_d$, with the distinction that when $d = k$, the basis of $Z_d$ consists of $k$-cycles.

We notice that the characteristic maps $\sigma^{(k)} \rightarrow X$ for all the $k$-simplices of $X$ provide us with a map $\Phi: \coprod_i(\sigma^{(k)}_i, \sigma^{(k-1)}i) \rightarrow (X^k, X^{k-1})$. It is evident that this map induces a homeomorphism $\Phi\star: \coprod_i \sigma^{(k)}_i/\coprod_i \sigma^{(k-1)}_i \rightarrow X^k/X^{k-1}$. As a result, we have $H_d(\coprod_i (\sigma^{(k)}_i, \sigma^{(k-1)}_i)) \cong H_d(X^k, X^{k-1})$.

Utilizing the Excision Theorem \ref{excisiontheorem}, which allows us to replace a subspace with its complement while preserving homology, we can conclude that there exists an isomorphism $H_d(X,A) \rightarrow H_d(X/A)$ for all good pairs $(X,A)$. Consequently, we have $H_d(\coprod_i (\sigma^{(k)}_i, \sigma^{(k-1)}_i)) \cong H_d(X^k, X^{k-1}) \cong H_d(X^k/X^{k-1})$. Through transitivity, this establishes $H_d(\coprod_i (\sigma^{(k)}_i, \sigma^{(k-1)}_i)) \cong H_d(X^k/X^{k-1})$. This result shows that $H_d(X^k, X^{k-1})$ is trivial for $d \neq k$ and is a free abelian group with the basis being the relative cycles defined by the maps $\sigma^{(k)} \rightarrow X$.

To complete the argument, we use induction and assume that the second and fifth parts of the homology long exact sequence are isomorphisms for dimensions less than $k$. In other words, we assume that:
\begin{itemize}
	\item $H_{d+1}^\Delta(X^{k-1}, X^{k-2}) \cong H_{d+1}(X^{k-1}, X^{k-2})$ for $d \leq k-2$.
	\item $H_d^\Delta(X^{k-1}, X^{k-2}) \rightarrow H_d(X^{k-1}, X^{k-2})$ is an isomorphism for $d \leq k-1$.
\end{itemize}
Now, we aim to show that the map $H_d^\Delta(X^k, X^{k-1}) \rightarrow H_d(X^k, X^{k-1})$ is an isomorphism for all $d$. We already know this holds for $d \neq k$. For $d = k$, we have:
\begin{equation}
H_k^\Delta(X^k,X^{k-1}) \cong H_k(X^k,X^{k-1})
\end{equation}
This follows from the Five Lemma \ref{fivelemma} regarding the isomorphism between the relative simplicial homology group $H_k^\Delta(X^k, X^{k-1})$ and the relative homology group $H_k(X^k, X^{k-1})$. Having established the isomorphism for all $d$, we've shown that the map 
$H_d^\Delta(X^k, X^{k-1}) \rightarrow H_d(X^k, X^{k-1})$ is indeed an isomorphism for all dimensions.

In summary, by induction and the previous arguments, we've demonstrated that for all $k$, the homomorphisms $H_d^\Delta(X^k, X^{k-1}) \rightarrow H_d(X^k, X^{k-1})$ are isomorphisms, which confirms the equivalence between simplicial and singular homology.
\end{proof}

\begin{thebibliography}{100}
\bibitem{1} Boissonnat, J. D., Chazal, F., Yvinec, M. (2018). Geometric and Topological Inference (Vol. 57). Cambridge University Press.
\bibitem{2} Edelsbrunner, H., Harer, J. L. (2022). Computational Topology: An Introduction. American Mathematical Society.
\bibitem{3} Hatcher, A. (2005). Algebraic Topology. Cambridge University Press.
\bibitem{4} Jonsson, J. (2011). Introduction to Simplicial Homology. Königliche Technische Hochschule. URL: \url{https://people.kth.se/~jakobj/doc/homology/homology.pdf}.
\bibitem{5} Khoury, M. (2022). Lecture 6: Introduction to Simplicial Homology. Topics in Computational Topology: An Algorithmic View. Ohio State University. URL: \url{http://web.cse.ohio-state.edu/~wang.1016/courses/788/Lecs/lec6-marc.pdf}.
\bibitem{6} Melodia, L., Lenz, R. (2021). Estimate of the Neural Network Dimension Using Algebraic Topology and Lie Theory. In Pattern Recognition. ICPR International Workshops and Challenges.
\bibitem{7} Nadathur, P. (2007). An Introduction to Homology. University of Chicago. URL: \url{https://www.math.uchicago.edu/~may/VIGRE/VIGRE2007/REUPapers/FINALFULL/Nadathur.pdf}.
\bibitem{8} Pontryagin L. S. (1952): Foundations of Combinatorial Topology. Graylock Press.
\end{thebibliography}
\end{document}