\documentclass{article}
\usepackage[utf8]{inputenc}
\usepackage[english]{babel}
\usepackage{amsthm,amssymb,amsmath}
\usepackage{titlesec}
\usepackage{url}
\usepackage[backend=biber]{biblatex}
\newtheorem{theorem}[section]{Theorem}
\newtheorem{definition}{Definition}[section]
\newtheorem{lemma}[definition]{Lemma}
\newtheorem{proposition}[definition]{Proposition}
\newtheorem{corollary}[definition]{Corollary}
\newtheorem{example}[definition]{Example}

\newcommand{\NN}{\mathbb{N}}
\newcommand{\ZZ}{\mathbb{Z}}
\newcommand{\RR}{\mathbb{R}}
\newcommand{\QQ}{\mathbb{Q}}
\newcommand{\CC}{\mathbb{C}}

\title{\textbf{Simplicial Homology} \\[0.2cm] \large An introduction to simplicial homology theory}

\author{Luciano Melodia}
\date{\today}

\begin{document}

\maketitle
\begin{abstract}
This paper explores the foundational concepts of simplicial structures that form the basis of simplicial homology theory. It also introduces singular homology as a means to establish the equivalence of homology groups for homeomorphic topological spaces. The paper concludes by providing a proof of the equivalence between simplicial and singular homology groups.
\end{abstract}
\tableofcontents

\section{Simplicial Complexes}
We would like to emphasize that a collection of points $X = \{x_0, x_1, \ldots, x_d\}$ in $\mathbb{R}^n$ is considered to be \emph{affinely independent} if these points do not lie within any affine subspace of dimension lower than $d$.

\begin{definition}
Given a set $X = \{x_0, x_1, \ldots, x_d\} \subset \mathbb{R}^n$ consisting of $d+1$ \textbf{affinely independent} points, the $d$-dimensional simplex $\sigma$, also known as a \textbf{\emph{$d$-simplex}}, is defined as the set of all convex combinations of these points.
\begin{equation}
	\sigma := \left\{\sum_{i=0}^{d} \lambda_i x_i \; \vert \; \sum_{i=0}^{d} \lambda_i = 1, \; \lambda_i \geq 0 \right\}.
\end{equation}
\end{definition}

As a convention, the empty set $\emptyset$ is included as a face, representing the simplex formed by the empty subset of vertices. A $0$-simplex represents a single point, a $1$-simplex represents a line segment connecting two points, a $2$-simplex represents a triangle, and a $3$-simplex represents a tetrahedron. It is worth mentioning that the $d$-simplex is homeomorphic to the $d$-dimensional disk $D^d$.

Furthermore, it is worth noting that $\sigma$ represents the convex hull of the points $X$, which can be defined as the smallest convex subset of $\mathbb{R}^n$ that contains all the points $x_0, x_1, \ldots, x_d$. The \emph{faces} of the simplex $\sigma$ with vertex set $X$ are simplices formed by subsets of $X$. An \emph{$d$-face} of a simplex refers to a subset of the vertices of the simplex with a cardinality of $d+1$. The faces of an $d$-simplex with a dimension less than $d$ are known as its \emph{proper faces}. Two simplices are considered to be \emph{properly situated} if their intersection is either empty or a face of both simplices. By identifying simplices along entire faces, we can construct the resulting \emph{simplicial complexes}.

\begin{definition}
A \textbf{simplicial complex} $K$ is a finite collection of simplices that satisfies the following properties:
\begin{enumerate}
	\item For every simplex $\sigma$ in $K$ and every face $\tau$ of $\sigma$, it follows that $\tau$ is also in $K$.
	\item If $\sigma$ and $\tau$ are both simplices in $K$, then they are properly situated.
\end{enumerate}
\end{definition}

The \emph{dimension} of $K$ is defined as the highest dimension among its simplices. For a simplicial complex $K$ in $\mathbb{R}^n$, its \emph{underlying space} $\vert K \vert$ is the union of all the simplices in $K$. The topology of $K$ is determined by the topology induced on $\vert K \vert$ by the standard topology in $\mathbb{R}^n$. It is important to note that when the vertex set is known, a simplicial complex in $\mathbb{R}^n$ can be fully characterized by listing its simplices. As a result, we can describe it purely in terms of combinatorics using \emph{abstract simplicial complexes}.

\begin{definition}
Consider a finite set $V = \{v_1, \ldots, v_n\}$. An \textbf{abstract simplicial complex $\tilde{K}$} with vertex set $V$ is a collection of finite subsets of $V$ that satisfies the following two conditions:
\begin{enumerate}
	\item All elements of $V$ are included in $\tilde{K}$.
	\item If $\sigma$ is a subset of $\tilde{K}$ and $\tau$ is a subset of $\sigma$, then $\tau$ is also a subset of $\tilde{K}$.
\end{enumerate}
\end{definition}

The abstract simplicial complex $\tilde{K}$ associated with a simplicial complex $K$ is commonly referred to as its \emph{vertex scheme}. Conversely, if an abstract complex $\tilde{K}$ serves as the vertex scheme for a complex $K$ in $\mathbb{R}^n$, then $K$ is known as a \emph{geometric realization} of $\tilde{K}$.

\begin{lemma}
Every finite abstract simplicial complex $\tilde{K}$ can be realized geometrically in a Euclidean space.
\end{lemma}

\begin{proof}
Let $\{v_1,v_2, \ldots, v_n\}$ denote the vertex set of $\tilde{K}$, where $n$ represents the number of vertices in $\tilde{K}$. Consider $\sigma \subset \mathbb{R}^n$, the simplex formed by the span of $\{e_1, e_2, \ldots, e_n\}$, where $e_i$ represents the $i$th unit vector. In this context, $K$ refers to the subcomplex of $\sigma$ such that $[e_{i_0}, \ldots, e_{i_d}]$ is a $d$-simplex of $K$ if and only if $[v_{i_0}, \ldots, v_{i_k}]$ is a simplex of $\tilde{K}$.
\end{proof}

\paragraph{Note:}
All realizations of an abstract simplicial complex are homeomorphic to each other. The specific realization mentioned above is referred to as the \emph{natural realization}. Furthermore, it has been proven that any finite abstract simplicial complex of dimension $n$ can be realized as a simplicial complex in $\mathbb{R}^{2d+1}$.

\section{Homology Groups}
Given a set $V$ representing the vertices of a simplex $\sigma$, we can establish an \emph{orientation} for the simplex by selecting a specific ordering for the vertices. If the vertex ordering differs from our chosen order by an odd permutation, it is considered \emph{reversed}, while even permutations are said to \emph{preserve} the orientation. Consequently, any simplex can have only two possible orientations. Moreover, the orientation of a $d$-simplex induces an orientation on its $(d-1)$-faces. To be more precise, if $\sigma^{(d)} := (v_0, v_1, \ldots, v_d)$ represents an oriented $d$-simplex, then the orientation of the $(d-1)$-face $\tau$ of $\sigma^{(d)}$ with the vertex set $\{v_0,\ldots,v_{i-1},v_{i+1},\ldots,v_d\}$ is given by $\tau_i = (-1)^i (v_0, \ldots,v_{i-1},v_{i+1},\ldots,v_d)$.

\begin{definition}
Given a set $\{\sigma_1^{(d)}, \ldots, \sigma_k^{(d)}\}$ of arbitrarily oriented $d$-simplices of a complex $K$ and an abelian group $G$, we define a \textbf{$d$-chain} $c$ with coefficients $g_i \in G$ as a formal sum.
\begin{equation}
c := g_1 \sigma^{(d)}_1 + g_2 \sigma^{(d)}_2 + \ldots + g_k \sigma^{(d)}_k = \sum_{i=1}^{k} g_i \sigma^{(d)}_i.
\end{equation}
\end{definition}

\paragraph{Note:} Henceforth we will assume that $G = (\mathbb{Z},+)$.

\begin{lemma}
The set of simplicial $d$-chains $L^\Delta_d$ is an abelian group $(L^\Delta_d,+)$.
\end{lemma}
\begin{proof}
The identity element of the group is represented by the empty chain $\sum_{i=1}^{k} e_G \sigma^{d}_i = e_G$. The sum of two chains is defined as $c+c' = \sum_{i=1}^{k} g_i \sigma_i^{(d)} + \sum_{j=1}^{l} g'_j \sigma_j^{(d)} = \sum_{i=1}^{k} (g_i+g_i') \sigma_i^{(d)} + \sum_{j=k}^{l} g'_j \sigma_j^{(d)}$ if $k \leq l$ and $c+c' = \sum_{i=1}^{k} g_i \sigma_i^{(d)} + \sum_{j=1}^{l} g'_j \sigma_j^{(d)} = \sum_{i=1}^{l} (g_i+g_i') \sigma_i^{(d)} + \sum_{j=l}^{k} g_j \sigma_j^{(d)}$ if $k > l$, thus, we can conclude that $c+c' \in L^\Delta_d$. The associativity of the group operation in $L^\Delta_d$ follows directly from the associativity of the group operation in $G$. The inverse element is defined by $e_{L^\Delta_d} = c + (-c) = \sum_{i=1}^{k} g_i \sigma_i^{(d)} - \sum_{i=1}^{k} (-g_i) \sigma_i^{(d)} = \sum_{i=1}^{k} (g_i-g_i) \sigma_i^{(d)}$ with $c,-c \in L^\Delta_d$.
\end{proof}

\begin{definition}
Let $\sigma^{(d)}$ be an oriented $d$-simplex in a complex $K$. The $\textbf{boundary}$ of $\sigma^{(d)}$ is defined as the simplicial $(d-1)$-chain of $K$ with coefficients in the abelian group $G = \mathbb{Z}$, given by
\begin{equation}
\partial(\sigma^{(d)}) = \sigma^{(d-1)}_0 + \sigma^{(d-1)}_1 + \ldots + \sigma^{(d-1)}_d = \sum_{i=1}^{d} \sigma^{(d-1)}_i
\end{equation}
where $\sigma^{(d-1)}_i$ is an $(d-1)$-face of $\sigma^{(d)}$. If $d=0$, we define $\partial(\sigma^{(0)}) = e_G = 0$.
\end{definition}

Since $\sigma^{(d)}$ is an oriented simplex, the $\sigma^{(d-1)}_i$-faces also have associated orientations. We can extend the definition of the boundary linearly to all elements of $L^\Delta_d$.

\begin{lemma}
The \textbf{boundary operator} is a group homomorphism $\partial: L^\Delta_d \rightarrow L^\Delta_{d-1}$.
\end{lemma}
\begin{proof}
We define the boundary operator for a $d$-chain $c = \sum_{i=1}^{k} g_i \sigma_i^{(d)}$ as follows: $\partial(c) = \sum_{i=1}^{k} g_i \partial(\sigma_i^{(d)}) = \sum_{i=1}^{k} g_i \sum_{j=1}^{d} \sigma_j^{(d-1)} = \sum_{i=1}^{k} \sum_{j=1}^{d} g_i \sigma_j^{(d-1)} \in L^\Delta_{d-1}$, where $\sigma_i^{(d)}$ are the $d$-simplices of $K$. We can compute this by
\begin{align}
\partial(c + c') &= \partial(\sum_{i=1}^{k} g_i \sigma_i^{(d)} + \sum_{j=1}^{l} g'_j \sigma_j^{(d)}) \\
&= \partial\left(\sum_{i=1}^{k} g_i \sigma_i^{(d)}\right) + \partial\left(\sum_{j=1}^{l} g'_j \sigma_j^{(d)}\right) \\
&= \sum_{i=1}^{k} g_i \partial(\sigma_i^{(d)}) + \sum_{j=1}^{l} g'_j \partial(\sigma_j^{(d)}) \\
&= \partial(c) + \partial(c').
\end{align}
\end{proof}

\begin{example}
Let's consider the $2$-simplex $\sigma^{(2)}$ with vertices $v_0$, $v_1$, and $v_2$. The $1$-faces of this simplex are $e_0 = (v_1,v_2)$ connecting $v_1$ and $v_2$, $e_1 = (v_2,v_0)$ connecting $v_2$ and $v_0$, and $e_2 = (v_0,v_1)$ connecting $v_0$ and $v_1$. Now, let's proceed with the computation.
\begin{align}
\partial(\partial(\sigma^{(2)})) &= \partial (e_1+e_2+e_3) \\
&= \partial e_1 + \partial e_2 + \partial e_3 \\
&= \partial(v_0,v_1) + \partial(v_1,v_2) + \partial(v_2,v_0) \\
&= [(v_1)-(v_0)] + [(v_2)-(v_1)]+[(v_0)-(v_2)].
\end{align}
We observe that $L^\Delta_0$ is an abelian group and that oppositely oriented simplices cancel each other out, resulting in $\partial(\partial(\sigma^{(2)})) = 0$. This property can be generalized to higher dimensions through induction. Therefore, since $\partial$ is a linear operator and the chain $c$ is a sum of $d$-simplices, we can conclude that $\partial^2(c) = 0$ for any $d$-chain $c$ in $L^\Delta_d$. Consequently, the boundary of the boundary is zero. Moreover, if the boundary of a simplex is zero, it is referred to as a \emph{cycle}. By this definition, we can deduce that the boundary of any simplex is a cycle.
\end{example}

\begin{definition}
A $d$-chain is referred to as a \textbf{cycle} if its boundary is equal to zero. We denote the set of $d$-cycles of a complex $K$ over the group $\mathbb{Z}$ as $Z_d$, the simplicial \textbf{cycle group}. It is important to note that $Z^\Delta_d$ is a subgroup of $L^\Delta_d$ and can also be expressed as $Z^\Delta_d = \ker(\partial)$.
\end{definition}

\begin{definition}
A $d$-cycle of a $k$-complex $K$ is said to be \textbf{homologous to zero} if it can be expressed as the boundary of an $(d+1)$-chain in $K$, where $d=0,1,\ldots,k-1$. In other words, a cycle is considered a boundary if it can be \glqq filled in\grqq{} by a higher-dimensional chain. This equivalence relation is denoted as $c \sim 0$, and the subgroup of $Z^\Delta_d$ consisting of boundaries is referred to as the simplicial \textbf{boundary group $B^\Delta_d$}. It is worth noting that $B^\Delta_d$ is equal to the image of the boundary operator $\partial$.
\end{definition}

Since $B^\Delta_d$ is a subgroup of $Z^\Delta_d$ and $Z^\Delta_d$ is an abelian group, every subgroup of $Z^\Delta_d$ is normal. Therefore, we can construct the quotient group $H^\Delta_d = Z^\Delta_d / B^\Delta_d$.

\begin{definition}
The group $H^\Delta_d$ represents the $d$-dimensional simplicial \textbf{homology group} of the complex $K$ over $\mathbb{Z}$. It can be expressed as the quotient group $\ker(\partial) / \text{im}(\partial)$.
\end{definition}

Next, we want to examine the structure of this homology group by shedding light on its connection to the connected components of a simplicial complex. We will find that the homology groups of the connected components of the complex, which in turn form a complex themselves, yield the direct sum of the homology group of the entire complex.

\begin{definition}
A \textbf{subcomplex} is defined as a subset $S$ of the simplices belonging to a complex $K$, where $S$ itself forms a complex.
\end{definition}

The collection of all simplices in a complex $K$ with dimensions less than or equal to $d$ is referred to as the $d$-skeleton of $K$. By definition, the $d$-skeleton forms a subcomplex.

\begin{definition}
A complex $K$ is considered \textbf{connected} if it cannot be expressed as the disjoint union of two or more non-empty subcomplexes. A geometric complex is \textbf{path-connected} if there exists a path consisting of $1$-simplices connecting any vertex to any other vertex.
\end{definition}

\begin{lemma}
\label{pathconnect}
Path-connectedness $\Longleftrightarrow$ connectedness.
\end{lemma}

\begin{proof}
\glqq $\Longrightarrow$\grqq{}: Let us assume that $K$ is not connected. In this case, we can choose two separate subcomplexes, namely $L$ and $M$, which do not share any common elements, but when combined, they form the entire complex $L \cap M = K$. Now, let's suppose that there exists a path between a vertex $l_0$ in $L$ and a vertex $m_0$ in $M$. However, if we consider the last vertex $l_i$ in this path that belongs to $L$, we observe that the $1$-simplex connecting $l_i$ to the next vertex in the path cannot be a part of either $L$ or $M$. If it were, then $L$ and $M$ would have a nonempty intersection, which contradicts our initial assumption that $K$ is not connected.

\glqq $\Longleftarrow$\grqq{}: Now, let's consider the other direction. Suppose there are two points, namely $l_0$ and $m_0$, in $K$ that do not have a path connecting them. In this case, we can define $L$ as the path-connected subcomplex of $K$ that contains $l_0$, and $M$ as the path-connected subcomplex that contains $m_0$. If there exists a vertex $v_0$ in the intersection of $L$ and $M$ (i.e., $v_0 \in L \cap M \neq \emptyset$), then we can find a path from $l_0$ to $v_0$ and another path from $v_0$ to $m_0$. By concatenating these paths, we obtain a path from $l_0$ to $m_0$, which contradicts our initial assumption that there is no path between $l_0$ and $m_0$. Therefore, we conclude that $L$ and $M$ must have an empty intersection ($L\cap M= \emptyset$), indicating that $K$ is not connected.
\end{proof}

\begin{theorem}
\label{decomptheorem}
Let $K_1, \ldots, K_p$ be the collection of all connected components of a complex $K$. Furthermore, let $H_{d_i}$ represent the $d$th homology group of $K_i$, and $H_d$ denote the $d$th homology group of $K$. In this context, we can establish that $H_d$ is isomorphic to the direct sum $H_{d_1} \oplus \cdots \oplus H_{d_p}$.
\end{theorem}

\begin{proof}
Let $L_d$ represent the group of $d$-chains of $K$, and $K_i$ denote the $i$th component of $K$. We can define $L_{d_i}$ as the group of $d$-chains of $K_i$. It is evident that $L_{d_i}$ is a subgroup of $L_d$. Furthermore, we observe that $L_d$ can be expressed as the direct sum of $L_{d_1}, \ldots, L_{d_p}$:
\begin{equation}
L_d = L_{d_1} \oplus \cdots \oplus L_{d_p}.
\end{equation}
Our goal is to demonstrate that a similar decomposition can be applied to the groups $B_d$ and $Z_d$. By considering $B_{d_i}$ as the image of $\partial$ restricted to the subgroup $L_{d_i}$, we can represent the group $B_d$ as the direct sum of these restrictions:
\begin{equation}
B_d = B_{d_1} \oplus \cdots \oplus B_{d_p}.
\end{equation}
Thus, for any element $c \in L_{d+1}$, which can be represented as:
\begin{align}
c = c_1 + \cdots + c_p, \quad \partial(c) = \partial c_1 + \cdots + \partial c_p \in B_d,
\end{align}
where $c_i \in L_{{(d+1)}_i}$.
Let us define $Z_{d_i}$ as the intersection of the kernel of $\partial$ and $L_{d_i}$. It follows that $Z_d$ can be expressed as the direct sum of $Z_{d_1}, \ldots, Z_{d_p}$:
\begin{equation}
Z_d = Z_{d_1} \oplus \cdots \oplus Z_{d_p}.
\end{equation}
To verify this, we observe that for an element $c \in L_d$ to belong to $Z_d$, we require $\partial(c) = 0$. However, we can express $\partial(c)$ as $\partial(c_1) + \cdots + \partial(c_p)$. Therefore, for $\partial(c) = 0$ to hold, it implies that $\partial(c_i) = 0$, indicating that $c_i \in Z_{d_i}$.
Since both $Z_d$ and $B_d$ can be decomposed componentwise, we can conclude that:
\begin{equation}
Z_d / B_d = Z_{d_1} / B_{d_1} \oplus \cdots \oplus Z_{d_p} / B_{d_p},
\end{equation}
and consequently:
\begin{equation}
H_d = H_{d_1} \oplus \cdots \oplus H_{d_p}.
\end{equation}
\end{proof}

\begin{definition}
The \textbf{index} of a chain $c = \sum_{i=1}^{k} g_i \sigma_i^{(n)}$ is defined as the sum of the coefficients $I(c) = \sum_{i=1}^{k} g_i$.
\end{definition}

\begin{proposition}
\label{decomposition}
If $K$ is a connected complex and $c$ is a $0$-chain with $I(c) = 0$, then the condition $I(c) = 0$ is equivalent to $c \sim 0$, where $\sim$ denotes homology equivalence. Furthermore, in this case, the zeroth simplicial homology group $H^\Delta_0(K,\mathbb{Z})$ is isomorphic to the integers $\mathbb{Z}$.
\end{proposition}

\begin{proof}
We begin by proving that $c \sim 0 \implies I(c) = 0$. Let $\sigma^{(1)} = (v_0,v_1)$ be a $1$-simplex. Then, for a chain $c = \partial(g\sigma^{(1)}) = gv_1-gv_0$, we have $c \sim 0$. It is clear that $I(c) = I(g\sigma^{(1)}) = g-g = 0$. Since $I(c+c') = I(c) + I(c')$, $I$ is a group homomorphism. For any $c \in L_1$ of the form $\sum_{i=1}^{k} g_i \sigma_i^{(1)}$, where $\sigma_i^{(1)} = (v_i,v_{i+1})$, we have $c = \partial(c) \sim 0 \implies I(c) = I(\partial(c)) = 0$.

To prove the forward direction, $I(x) = 0 \implies c \sim 0$, we consider two vertices $v$ and $w$ of $K$. Since $K$ is connected, there exists a path between $v$ and $w$ consisting of $1$-simplices $\sigma_i^{(1)} = (v_i,v_{i+1})$, $i=1,\ldots,k-1$, where $v_0 = v$ and $v_k = w$. We consider the boundary of the chain $c = \sum_{i=1}^{k} g \sigma_i^{(1)}$, given by $\partial(c) = \sum_{i=1}^{k}g \partial(\sigma_i^{(1)}) = \sum_{i=1}^{k}g [(v_{i+1}) - (v_i))] = gw - gv$. The index of the chain $c = \sum_{i=1}^{k} g_i \sigma_i^{(n)}$ is defined as $I(c) = \sum_{i=1}^{k} g_i$. Since $\partial(c)$ is a boundary, we have $c = \partial(c) \sim 0$. This implies that $(gw-gv) \sim 0$, which further implies $gw \sim gv$. Therefore, any $0$-chain $c$ in $K$ is homologous to the chain $gv$. We observe that homologous chains have equal indices, i.e., $I(c) = I(gv) = g$. Thus, we have $c \sim gv \implies c \sim I(c)v$. This shows that if $I(c) = 0$, then $c \sim 0$. Hence, $I(c) = 0$ is equivalent to $c \sim 0$.

As mentioned, $I$ is a homomorphism from $L_0 = Z_0$ to $\mathbb{Z}$. For a $0$-simplex $c$ and $g \in \mathbb{Z}$, the chain $gc \in L_0$ is a cycle with $I(gc) = g$. Therefore, $I(Z_0) = \mathbb{Z}$. Since $I(c) = 0$ is equivalent to $c \sim 0$, we have $B_0 = \ker(I)$. This implies that $H^\Delta_0 = Z_0/B_0 \cong \mathbb{Z}$.
\end{proof}

At this point, we can deduce the following corollary from Theorem \ref{pathconnect} and Proposition \ref{decomposition}:

\begin{corollary}
\label{directsum0hom}
The zero-dimensional homology group of a complex $K$ over $\mathbb{Z}$ can be represented as $\mathbb{Z}^p = \bigoplus_p \mathbb{Z}$, where $p$ denotes the number of connected components present in $K$.
\end{corollary}

\begin{example}
\begin{itemize}
	\item[]
	\item This implies that the zeroth homology group of the circle is isomorphic to $\mathbb{Z}$. If we consider a simplicial representation of the circle using four one-simplices, $v_1 = (w,v)$, $v_2 = (v,y)$, $v_3 = (y,x)$, and $v_4 = (x,w)$, the group $Z_0$ consists of sums over the four zero-simplices $v$, $w$, $x$, and $y$ with coefficients in $\mathbb{Z}$. Let $c$ be a zero-chain with non-zero coefficients given by
	\begin{equation}
	c = g_1v+g_2w+g_3x+g_4y.
	\end{equation}
	In order to reduce it to an element of $H^\Delta_0$, we subtract from it the chain $c' = g_4x-g_4y \sim 0$, resulting in
	\begin{equation}
	c-c' = g_1v+g_2w +(g_3-g_4)x.
	\end{equation}
	By repeating this process, we obtain a new chain
	\begin{equation}
	c'' = (g_1-g_2+g_3-g_4)v.
	\end{equation}
	Since $c'' \sim c$, it represents an element of $H^\Delta_0$. Moreover, since $g_i \in \mathbb{Z}$, we can write $(g_1-g_2+g_3-g_4) \in \mathbb{Z}$ as $c'' = gv$, where $g$ is an element of $\mathbb{Z}$. Therefore, we can choose any $g$, which implies that $H^\Delta_0 \cong \mathbb{Z}$.
	\item We will demonstrate that $H^\Delta_n(S^n) \cong \mathbb{Z}$. It is worth noting that the $n$-simplex $\sigma^{(n)}$ and the $n$-ball are homeomorphic. Consequently, their boundaries, which consist of $(n-1)$-simplices, and the $n$-sphere are also homeomorphic. Therefore, the appropriate simplicial structure to impose on $S^n$ is that of the boundary of the $(n+1)$-simplex $\sigma^{(n+1)}$. Let ${v_0,\ldots,v_{n-1}}$ denote the set of vertices of $\sigma^{(n+1)}$. It is important to note that this set is not oriented, and the orientations of the $(n-1)$-simplices can be arbitrarily determined. We will utilize their numbering to establish orientations. Consequently, all $n$-chains on this structure can be expressed as:
	\begin{equation}
	\label{chain}
	c = \sum_{i=0}^{n+1}g_i (v_0,\ldots,v_{i-1},v_{i+1},\ldots,v_n),
	\end{equation}
	where $g_i \in \mathbb{Z}$. Since $\sigma^{(n+1)}$ itself is not part of the structure, there are no boundaries in $Z_n$, the group of cycles. Therefore, $H_n = Z_n/B_n$ represents the group of cycles. If $c \in Z_n$, then $\partial(c) = 0$. By using Eq. \ref{chain}, we have:
	\begin{align} \partial(c) = &\partial\left(\sum_{i=0}^{n+1} g_i(v_0,\ldots,v_{i-1},v_{i+1},\ldots,v_n) \right) \\ = &\sum_{i=0}^{n+1}g_i \big(\sum_{j1}^{n+1}(-1)^j(v_0,\ldots,v_{i-1},v_{i+1},\ldots,v_{j-1},v_{j+1},\ldots,v_n)\big). \end{align}
	By rearranging this sum, we obtain terms of the form:
	\begin{equation} \label{terms} (g_k-g_l)(v_0,\ldots,v_{j-1},v_{j+1},\ldots,v_{i-1},v_{i+1},\ldots,v_n) \end{equation}
	where $k,l = 0, \ldots, n+1$ for all $i,j = 0, \ldots, n$. Each pair of $n$-simplices of $\sigma^{(n+1)}$ intersect along an $(n-1)$-face. Therefore, we obtain terms of the form given in Eq. \ref{terms} for each of these faces. From this, we can deduce that if $\partial(c) = 0$, we must have $g_k = g_l$ for all $k,l = 0, \ldots, n+1$. In other words, $g_0 = g_1 = \cdots = g_{n+1}$. Consequently, our original $n$-chain can be rewritten as:
	\begin{equation} c = \sum_{i=0}^{n+1}g_0(v_0,\ldots,v_{i-1},v_{i+1},\ldots,v_n), \end{equation}
	allowing us to choose $g_0$ from $\mathbb{Z}$. Thus, we conclude that $H^\Delta_n(S^n) \cong \mathbb{Z}$.
	\item We demonstrate that $H^\Delta_n(D^n) = 0$. To do so, we employ the simplest simplicial structure for $D^n$, which is that of the $n$-simplex $\sigma^{(n)}$. Consequently, all $n$-chains can be expressed as $c = g \sigma^{(n)}$, where $g \in \mathbb{Z}$. It is important to note that this form is never a boundary, thus implying that $H_n = Z_n$. However, $\partial(c) = 0$ only when $g = 0$. Consequently, we can conclude that $H^\Delta_n(D^n) \cong 0$.
\end{itemize}
\end{example}

\section{Singular Homology}
In the realm of lower dimensions, we possess an intuitive understanding of when two topological spaces are fundamentally \glqq equivalent\grqq{}. To formalize and solidify this intuition, we have devised various methods, one of which is the concept of homeomorphism. It would be highly desirable to establish a relationship between the homology groups of homeomorphic spaces. Remarkably, it has been discovered that if two topological spaces are homeomorphic, their homology groups are isomorphic. This fact begs for verification.

To accomplish this task, we require a means of comparing homology groups. However, it is not immediately evident how we can achieve this using the tools we have developed thus far. In fact, it proves to be quite a challenging problem. To circumvent this difficulty, we introduce the notion of \emph{singular homology}. The fundamental principles underlying this concept are analogous to those we have already explored.
To compute $Z_n(X)$, we need to find the group of $n$-cycles in $X$. Since $X$ is obtained by identifying opposite faces of $\partial \sigma^{(n)}$, an $n$-cycle in $X$ corresponds to an $n$-cycle in $\partial \sigma^{(n)}$ that is not a boundary of any $(n+1)$-dimensional simplex in $\sigma^{(n)}$. In other words, an $n$-cycle in $X$ corresponds to an $n$-cycle in $\partial \sigma^{(n)}$ that is not a boundary of any $(n+1)$-dimensional face of $\sigma^{(n)}$.
\begin{definition}
In the context of a topological space $X$, we define a \textbf{singular $n$-simplex} as a map $\tilde{\sigma}^{(n)}: \sigma^{(n)} \rightarrow X$, where $\tilde{\sigma}^{(n)}$ is continuous.
\end{definition}

We define the boundary map $\partial_n$ in a similar manner as before:

\begin{definition}
The boundary map, denoted as $\partial_n$, is a function that operates on the chain group $C_n(X)$ and maps it to the chain group $C_{n-1}(X)$. It is defined as follows: For any singular $n$-simplex $\tilde{\sigma}^{(n)}$ in $X$, the boundary map $\partial_n(\tilde{\sigma}^{(n)})$ is obtained by summing over all the $(n-1)$-simplices that are obtained by removing one vertex from $\tilde{\sigma}^{(n)}$. Each term in the sum is multiplied by $(-1)^i$, where $i$ represents the index of the removed vertex. In other words, if $v_i$ represents the $0$-simplex (vertex) of $\tilde{\sigma}^{(n)}$, then the boundary map can be expressed as:
\begin{equation}
\partial_n(\tilde{\sigma}^{(n)}) = \sum_{i} (-1)^i \tilde{\sigma}^{(n)}\vert_{[v_0,\ldots,v_{i-1},v_{i+1},\ldots,v_n]}.
\end{equation}
Here, $v_i$ is a map that takes the $0$-simplex $\sigma^{(0)}$ to the corresponding vertex in $X$, s.t. $v_i: \sigma^{(0)} \rightarrow X$ is continuous.
\end{definition}

As mentioned earlier, when we apply the boundary map twice to an $n$-chain $c$, denoted as $\partial^2(c)$ or $\partial(\partial(c))$, the result is always zero. This observation leads us to the idea of defining the singular homology groups in a similar way to the simplicial homology groups.

\begin{definition}
The \textbf{singular homology group} $H_n(X)$ is defined to be the quotient $H_n(X) = \ker(\partial_n) / \text{im}(\partial_{n+1})$.
\end{definition}

In the following section, we will explore how this definition of homology allows us to establish a simple relationship between homeomorphic spaces and their corresponding homology groups. This relationship becomes apparent when we consider the fact that the definitions of $H_d$ and $H^\Delta_d$ are analogous. Intuitively, we would expect these two groups to be the same. However, this is not immediately obvious. One reason for this is that $H^\Delta_d$ is finitely generated, while the chain group $C_d(X)$, from which we derived $H_d$, is uncountable.

Interestingly, for spaces where both simplicial and singular homology groups can be calculated, these two groups are indeed equivalent. We will provide a proof for this later on. But before we do, let us present some facts about singular homology that support the intuition that $H_d$ is isomorphic to $H^\Delta_d$.

\begin{proposition}
In the context of a topological space $X$, it can be observed that $H_d(X)$ is isomorphic to the direct sum $H_d(X_1) \oplus \cdots \oplus H_d(X_p)$, where $X_i$ represents the path-connected components of $X$. This equivalence serves as the counterpart to Theorem \ref{decomptheorem}.\end{proposition}

\begin{proof}
As the maps $\tilde{\sigma}^{(d)}$ exhibit continuity, it can be deduced that a singular simplex always possesses a path-connected image within $X$. Consequently, $C_d(X)$ can be expressed as the direct sum of subgroups $C_d(X_1) \oplus \cdots \oplus C_d(X_p)$. The boundary map $\partial$ functions as a homomorphism, thereby preserving this decomposition. Consequently, $\ker(\partial_d)$ and $\text{im}(\partial_{d+1})$ also undergo a split, leading to the conclusion that $H_d(X) \cong H_d(X_1) \oplus H_d(X_2) \oplus \cdots \oplus H_d(X_p)$.
\end{proof}


\begin{proposition}
The zero-dimensional homology group of a space $X$ can be expressed as the direct sum of $\mathbb{Z}$ copies, with each copy corresponding to a distinct path-component of $X$. This correspondence serves as the parallel to Corollary \ref{directsum0hom}.\end{proposition}

\begin{proof}
To establish the isomorphism $H_0(X) \cong \mathbb{Z}$, it is sufficient to consider the case where $X$ is path-connected. For a $0$-chain $c$, the boundary operator $\partial_0(c)$ is always zero since the boundary of any $0$-simplex vanishes. Consequently, $\ker(\partial_0) = C_0(X)$, which implies that $H_0(X) = C_0(X) / \text{im}(\partial_1)$ by definition.

Let us define the map $I: C_0(X) \rightarrow \mathbb{Z}$, where $I(c) = \sum_i g_i$ for $c = \sum_i g_i \tilde{\sigma}^{(0)} \in C_0(X)$. Our goal is to demonstrate that $\ker(I) = \text{im}(\partial_1)$, or in other words, for any $0$-chain $c$, $I(c) = 0$ if and only if $c \sim 0$. The proof follows a similar line of reasoning as Proposition \ref{decomposition}.
\end{proof}

\begin{example}
Alternative proof that $H_d(S^d) \cong \mathbb{Z}$. To prove that the $d$th homology group of the $d$-sphere is isomorphic to $\mathbb{Z}$, we will use the singular homology approach. The $d$th singular chain group $C_d(S^d)$ consists of formal linear combinations of singular $d$-simplices in $S^d$ with integer coefficients. First, we note that $S^d$ is a connected and compact topological space. Therefore, by the Hurewicz theorem, we have $H_d(S^d) \cong \pi_d(S^d)$, where $\pi_d(S^d)$ denotes the $d$th homotopy group of $S^d$. Since $S^d$ is simply connected for $d \geq 2$, we have $\pi_d(S^d) = 0$ for $d \geq 2$. However, for $d = 1$, we have $\pi_1(S^1) \cong \mathbb{Z}$.

Now, we need to establish the isomorphism between $\pi_1(S^1)$ and $H_1(S^1)$. To do this, we consider the singular $1$-chain group $C_1(S^1)$, which consists of formal linear combinations of singular $1$-simplices in $S^1$ with integer coefficients. Let $c$ be a singular $1$-chain in $C_1(S^1)$. We can write $c$ as $c = \sum_{i} g_i \tilde{\sigma}^{(1)}_i$, where $g_i \in \mathbb{Z}$ and $\tilde{\sigma}^{(1)}_i$ are singular $1$-simplices. The boundary of $c$ is given by $\partial(c) = \sum_{i} g_i \partial(\tilde{\sigma}^{(1)}_i)$. Since $S^1$ is a $1$-dimensional manifold, the boundary of any singular $1$-simplex $\tilde{\sigma}^{(1)}_i$ is a formal linear combination of two points in $S^1$, each with opposite orientations. Therefore, $\partial(\tilde{\sigma}^{(1)}_i) = p - q$, where $p$ and $q$ are points in $S^1$. Hence, we have $\partial(c) = \sum_{i} g_i (p - q) = (p - q) \sum_{i} g_i$. Since $p$ and $q$ are fixed points in $S^1$, the sum $\sum_{i} g_i$ is an integer. Therefore, the boundary of any singular $1$-chain $c$ in $C_1(S^1)$ is of the form $(p - q)k$, where $k$ is an integer. This implies that $H_1(S^1) = Z_1(S^1) / B_1(S^1) \cong \mathbb{Z}$, where $Z_1(S^1)$ is the group of $1$-cycles and $B_1(S^1)$ is the group of $1$-boundaries. In conclusion, we have shown that $H_d(S^d) \cong \pi_d(S^d) = 0$ for $d \geq 2$, and $H_1(S^1) \cong \pi_1(S^1) \cong \mathbb{Z}$. Therefore, the $d$th homology group of the $d$-sphere is isomorphic to $\mathbb{Z}$. $\qed$
\end{example}

\section{Chain Complexes}
\section{Exact Sequences}
\section{Relative Homology Groups}
\section{Equivalence of Simplicial Homology Group $H_d^\Delta$ \\ and Singular Homology Group $H_d$}

\begin{thebibliography}{100}
\bibitem{1} Boissonnat, J. D., Chazal, F., Yvinec, M. (2018). Geometric and Topological Inference (Vol. 57). Cambridge University Press.
\bibitem{2} Edelsbrunner, H., Harer, J. L. (2022). Computational Topology: An Introduction. American Mathematical Society.
\bibitem{3} Hatcher, A. (2005). Algebraic Topology. Cambridge University Press.
\bibitem{4} Jonsson, J. (2011). Introduction to Simplicial Homology. Königliche Technische Hochschule. URL: \url{https://people.kth.se/~jakobj/doc/homology/homology.pdf}.
\bibitem{5} Khoury, M. (2022). Lecture 6: Introduction to Simplicial Homology. Topics in Computational Topology: An Algorithmic View. Ohio State University. URL: \url{http://web.cse.ohio-state.edu/~wang.1016/courses/788/Lecs/lec6-marc.pdf}.
\bibitem{6} Melodia, L., Lenz, R. (2021). Estimate of the Neural Network Dimension Using Algebraic Topology and Lie Theory. In Pattern Recognition. ICPR International Workshops and Challenges.
\bibitem{7} Nadathur, P. (2007). An Introduction to Homology. University of Chicago. URL: \url{https://www.math.uchicago.edu/~may/VIGRE/VIGRE2007/REUPapers/FINALFULL/Nadathur.pdf}.
\bibitem{8} Pontryagin L. S. (1952): Foundations of Combinatorial Topology. Graylock Press.
\end{thebibliography}
\end{document}