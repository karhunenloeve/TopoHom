\documentclass{article}
\usepackage[utf8]{inputenc}
\usepackage[english]{babel}
\usepackage{amsthm,amssymb,amsmath}
\usepackage{titlesec}
\usepackage{url}
\usepackage[backend=biber]{biblatex}
\newtheorem{theorem}[section]{Theorem}
\newtheorem{definition}{Definition}[section]
\newtheorem{lemma}[definition]{Lemma}
\newtheorem{example}[definition]{Example}

\newcommand{\NN}{\mathbb{N}}
\newcommand{\ZZ}{\mathbb{Z}}
\newcommand{\RR}{\mathbb{R}}
\newcommand{\QQ}{\mathbb{Q}}
\newcommand{\CC}{\mathbb{C}}

\title{\textbf{Simplicial Homology} \\[0.2cm] \large An introduction to simplicial homology theory}

\author{Luciano Melodia}
\date{\today}

\begin{document}

\maketitle
\begin{abstract}
This paper explores the foundational concepts of simplicial structures that form the basis of simplicial homology theory. It also introduces singular homology as a means to establish the equivalence of homology groups for homeomorphic topological spaces. The paper concludes by providing a proof of the equivalence between simplicial and singular homology groups.
\end{abstract}
\tableofcontents

\section{Simplicial Complexes}
We would like to emphasize that a collection of points $X = \{x_0, x_1, \ldots, x_d\}$ in $\mathbb{R}^n$ is considered to be \emph{affinely independent} if these points do not lie within any affine subspace of dimension lower than $d$.

\begin{definition}
Given a set $X = \{x_0, x_1, \ldots, x_d\} \subset \mathbb{R}^n$ consisting of $d+1$ \textbf{affinely independent} points, the $d$-dimensional simplex $\sigma$, also known as a \textbf{\emph{$d$-simplex}}, is defined as the set of all convex combinations of these points.
\begin{equation}
	\sigma := \left\{\sum_{i=0}^{d} \lambda_i x_i \; \vert \; \sum_{i=0}^{d} \lambda_i = 1, \; \lambda_i \geq 0 \right\}.
\end{equation}
\end{definition}

As a convention, the empty set $\emptyset$ is included as a face, representing the simplex formed by the empty subset of vertices. A $0$-simplex represents a single point, a $1$-simplex represents a line segment connecting two points, a $2$-simplex represents a triangle, and a $3$-simplex represents a tetrahedron. It is worth mentioning that the $d$-simplex is homeomorphic to the $d$-dimensional disk $D^d$.

Furthermore, it is worth noting that $\sigma$ represents the convex hull of the points $X$, which can be defined as the smallest convex subset of $\mathbb{R}^n$ that contains all the points $x_0, x_1, \ldots, x_d$. The \emph{faces} of the simplex $\sigma$ with vertex set $X$ are simplices formed by subsets of $X$. An \emph{$d$-face} of a simplex refers to a subset of the vertices of the simplex with a cardinality of $d+1$. The faces of an $d$-simplex with a dimension less than $d$ are known as its \emph{proper faces}. Two simplices are considered to be \emph{properly situated} if their intersection is either empty or a face of both simplices. By identifying simplices along entire faces, we can construct the resulting \emph{simplicial complexes}.

\begin{definition}
A \textbf{simplicial complex} $K$ is a finite collection of simplices that satisfies the following properties:
\begin{enumerate}
	\item For every simplex $\sigma$ in $K$ and every face $\tau$ of $\sigma$, it follows that $\tau$ is also in $K$.
	\item If $\sigma$ and $\tau$ are both simplices in $K$, then they are properly situated.
\end{enumerate}
\end{definition}

The \emph{dimension} of $K$ is defined as the highest dimension among its simplices. For a simplicial complex $K$ in $\mathbb{R}^n$, its \emph{underlying space} $\vert K \vert$ is the union of all the simplices in $K$. The topology of $K$ is determined by the topology induced on $\vert K \vert$ by the standard topology in $\mathbb{R}^n$. It is important to note that when the vertex set is known, a simplicial complex in $\mathbb{R}^n$ can be fully characterized by listing its simplices. As a result, we can describe it purely in terms of combinatorics using \emph{abstract simplicial complexes}.

\begin{definition}
Consider a finite set $V = \{v_1, \ldots, v_n\}$. An \textbf{abstract simplicial complex $\tilde{K}$} with vertex set $V$ is a collection of finite subsets of $V$ that satisfies the following two conditions:
\begin{enumerate}
	\item All elements of $V$ are included in $\tilde{K}$.
	\item If $\sigma$ is a subset of $\tilde{K}$ and $\tau$ is a subset of $\sigma$, then $\tau$ is also a subset of $\tilde{K}$.
\end{enumerate}
\end{definition}

The abstract simplicial complex $\tilde{K}$ associated with a simplicial complex $K$ is commonly referred to as its \emph{vertex scheme}. Conversely, if an abstract complex $\tilde{K}$ serves as the vertex scheme for a complex $K$ in $\mathbb{R}^n$, then $K$ is known as a \emph{geometric realization} of $\tilde{K}$.

\begin{lemma}
Every finite abstract simplicial complex $\tilde{K}$ can be realized geometrically in a Euclidean space.
\end{lemma}

\begin{proof}
Let ${v_1,v_2, \ldots, v_n}$ denote the vertex set of $\tilde{K}$, where $n$ represents the number of vertices in $\tilde{K}$. Consider $\sigma \subset \mathbb{R}^n$, the simplex formed by the span of ${e_1, e_2, \ldots, e_n}$, where $e_i$ represents the $i$th unit vector. In this context, $K$ refers to the subcomplex of $\sigma$ such that $[e_{i_0}, \ldots, e_{i_d}]$ is a $d$-simplex of $K$ if and only if $[v_{i_0}, \ldots, v_{i_k}]$ is a simplex of $\tilde{K}$.
\end{proof}

\paragraph{Note:}
All realizations of an abstract simplicial complex are homeomorphic to each other. The specific realization mentioned above is referred to as the \emph{natural realization}. Furthermore, it has been proven that any finite abstract simplicial complex of dimension $n$ can be realized as a simplicial complex in $\mathbb{R}^{2d+1}$.

\section{Homology Groups}
Given a set $V$ representing the vertices of a simplex $\sigma$, we can establish an \emph{orientation} for the simplex by selecting a specific ordering for the vertices. If the vertex ordering differs from our chosen order by an odd permutation, it is considered \emph{reversed}, while even permutations are said to \emph{preserve} the orientation. Consequently, any simplex can have only two possible orientations. Moreover, the orientation of a $d$-simplex induces an orientation on its $(d-1)$-faces. To be more precise, if $\sigma^{(d)} := (v_0, v_1, \ldots, v_d)$ represents an oriented $d$-simplex, then the orientation of the $(d-1)$-face $\tau$ of $\sigma^{(d)}$ with the vertex set $\{v_0,\ldots,v_{i-1},v_{i+1},\ldots,v_d\}$ is given by $\tau_i = (-1)^i (v_0, \ldots,v_{i-1},v_{i+1},\ldots,v_d)$.

\begin{definition}
Given a set $\{\sigma_1^{(d)}, \ldots, \sigma_k^{(d)}\}$ of arbitrarily oriented $d$-simplices of a complex $K$ and an abelian group $G$, we define a \textbf{$d$-chain} $c$ with coefficients $g_i \in G$ as a formal sum.
\begin{equation}
c := g_1 \sigma^{(d)}_1 + g_2 \sigma^{(d)}_2 + \ldots + g_k \sigma^{(d)}_k = \sum_{i=1}^{k} g_i \sigma^{(d)}_i.
\end{equation}
\end{definition}

\paragraph{Note:} Henceforth we will assume that $G = (\mathbb{Z},+)$.

\begin{lemma}
The set of $d$-chains $L_d$ is an abelian group $(L_d,+)$.
\end{lemma}
\begin{proof}
The identity element of the group is represented by the empty chain $\sum_{i=1}^{k} e_G \sigma^{d}_i = e_G$. The sum of two chains is defined as $c+c' = \sum_{i=1}^{k} g_i \sigma_i^{(d)} + \sum_{j=1}^{l} g'_j \sigma_j^{(d)} = \sum_{i=1}^{k} (g_i+g_i') \sigma_i^{(d)} + \sum_{j=k}^{l} g'_j \sigma_j^{(d)}$ if $k \leq l$ and $c+c' = \sum_{i=1}^{k} g_i \sigma_i^{(d)} + \sum_{j=1}^{l} g'_j \sigma_j^{(d)} = \sum_{i=1}^{l} (g_i+g_i') \sigma_i^{(d)} + \sum_{j=l}^{k} g_j \sigma_j^{(d)}$ if $k > l$, thus, we can conclude that $c+c' \in L_d$. The associativity of the group operation in $L_d$ follows directly from the associativity of the group operation in $G$. The inverse element is defined as $e_{L_d} = c + (-c) = \sum_{i=1}^{k} g_i \sigma_i^{(d)} - \sum_{i=1}^{k} (-g_i) \sigma_i^{(d)} = \sum_{i=1}^{k} (g_i-g_i) \sigma_i^{(d)}$ with $c,-c \in L_d$.
\end{proof}

\begin{definition}
Let $\sigma^{(d)}$ be an oriented $d$-simplex in a complex $K$. The $\textbf{boundary}$ of $\sigma^{(d)}$ is defined as the $(d-1)$-chain of $K$ with coefficients in the abelian group $G = \mathbb{Z}$, given by
\begin{equation}
\partial(\sigma^{(d)}) = \sigma^{(d-1)}_0 + \sigma^{(d-1)}_1 + \ldots + \sigma^{(d-1)}_d = \sum_{i=1}^{d} \sigma^{(d-1)}_i
\end{equation}
where $\sigma^{(d-1)}_i$ is an $(d-1)$-face of $\sigma^{(d)}$. If $d=0$, we define $\partial(\sigma^{(0)}) = e_G = 0$.
\end{definition}

Since $\sigma^{(d)}$ is an oriented simplex, the $\sigma^{(d-1)}_i$-faces also have associated orientations. We can extend the definition of the boundary linearly to all elements of $L_d$.

\begin{lemma}
The \textbf{boundary operator} is a group homomorphism $\partial: L_d \rightarrow L_{d-1}$.
\end{lemma}
\begin{proof}
We define the boundary operator for a $d$-chain $c = \sum_{i=1}^{k} g_i \sigma_i^{(d)}$ as follows: $\partial(c) = \sum_{i=1}^{k} g_i \partial(\sigma_i^{(d)}) = \sum_{i=1}^{k} g_i \sum_{j=1}^{d} \sigma_j^{(d-1)} = \sum_{i=1}^{k} \sum_{j=1}^{d} g_i \sigma_j^{(d-1)} \in L_{d-1}$, where $\sigma_i^{(d)}$ are the $d$-simplices of $K$. We can compute this by
\begin{align}
\partial(c + c') &= \partial(\sum_{i=1}^{k} g_i \sigma_i^{(d)} + \sum_{j=1}^{l} g'_j \sigma_j^{(d)}) \\
&= \partial\left(\sum_{i=1}^{k} g_i \sigma_i^{(d)}\right) + \partial\left(\sum_{j=1}^{l} g'_j \sigma_j^{(d)}\right) \\
&= \sum_{i=1}^{k} g_i \partial(\sigma_i^{(d)}) + \sum_{j=1}^{l} g'_j \partial(\sigma_j^{(d)}) \\
&= \partial(c) + \partial(c').
\end{align}
\end{proof}

\begin{example}
Let's consider the $2$-simplex $\sigma^{(2)}$ with vertices $v_0$, $v_1$, and $v_2$. The $1$-faces of this simplex are $e_0 = (v_1,v_2)$ connecting $v_1$ and $v_2$, $e_1 = (v_2,v_0)$ connecting $v_2$ and $v_0$, and $e_2 = (v_0,v_1)$ connecting $v_0$ and $v_1$. Now, let's proceed with the computation.
\begin{align}
\partial(\partial(\sigma^{(2)})) &= \partial (e_1+e_2+e_3) \\
&= \partial e_1 + \partial e_2 + \partial e_3 \\
&= \partial(v_0,v_1) + \partial(v_1,v_2) + \partial(v_2,v_0) \\
&= [(v_1)-(v_0)] + [(v_2)-(v_1)]+[(v_0)-(v_2)].
\end{align}
We observe that $L_0$ is an abelian group and that oppositely oriented simplices cancel each other out, resulting in $\partial(\partial(\sigma^{(2)})) = 0$. This property can be generalized to higher dimensions through induction. Therefore, since $\partial$ is a linear operator and the chain $c$ is a sum of $d$-simplices, we can conclude that $\partial^2(c) = 0$ for any $d$-chain $c$ in $L_d$. Consequently, the boundary of the boundary is zero. Moreover, if the boundary of a simplex is zero, it is referred to as a \emph{cycle}. By this definition, we can deduce that the boundary of any simplex is a cycle.
\end{example}

\begin{definition}
A $d$-chain is referred to as a \textbf{cycle} if its boundary is equal to zero. We denote the set of $d$-cycles of a complex $K$ over the group $\mathbb{Z}$ as $Z_d$, the \textbf{cycle group}. It is important to note that $Z_d$ is a subgroup of $L_d$ and can also be expressed as $Z_d = \ker(\partial)$.
\end{definition}

\begin{definition}
A $d$-cycle $o$ of a $k$-complex $K$ is said to be \textbf{homologous to zero} if it can be expressed as the boundary of an $(d+1)$-chain in $K$, where $d=0,1,\ldots,k-1$. In other words, a cycle is considered a boundary if it can be "filled in" by a higher-dimensional chain. This equivalence relation is denoted as $x \sim 0$, and the subgroup of $Z_d$ consisting of boundaries is referred to as the \textbf{boundary group $B_d$}. It is worth noting that $B_d$ is equal to the image of the boundary operator $\partial$.
\end{definition}

Since $B_d$ is a subgroup of $Z_d$ and $Z_d$ is an abelian group, every subgroup of $Z_d$ is normal. Therefore, we can construct the quotient group $H_d = Z_d / B_d$.

\begin{definition}
The group $H_d$ represents the $d$-dimensional \textbf{homology group} of the complex $K$ over $\mathbb{Z}$. It can be expressed as the quotient group $\ker(\partial) / \text{im}(\partial)$.
\end{definition}

Next, we want to examine the structure of this homology group by shedding light on its connection to the connected components of a simplicial complex. We will find that the homology groups of the connected components of the complex, which in turn form a complex themselves, yield the direct sum of the homology group of the entire complex.

\begin{definition}
A \textbf{subcomplex} is defined as a subset $S$ of the simplices belonging to a complex $K$, where $S$ itself forms a complex.
\end{definition}

The collection of all simplices in a complex $K$ with dimensions less than or equal to $d$ is referred to as the $d$-skeleton of $K$. By definition, the $d$-skeleton forms a subcomplex.

\begin{definition}
A complex $K$ is considered \textbf{connected} if it cannot be expressed as the disjoint union of two or more non-empty subcomplexes. A geometric complex is \textbf{path-connected} if there exists a path consisting of $1$-simplices connecting any vertex to any other vertex.
\end{definition}

\begin{lemma}
Path-connectedness $\Longleftrightarrow$ connectedness.
\end{lemma}

\begin{proof}
"$\Longrightarrow$": Let us assume that $K$ is not connected. In this case, we can choose two separate subcomplexes, namely $L$ and $M$, which do not share any common elements, but when combined, they form the entire complex $L \cap M = K$. Now, let's suppose that there exists a path between a vertex $l_0$ in $L$ and a vertex $m_0$ in $M$. However, if we consider the last vertex $l_i$ in this path that belongs to $L$, we observe that the $1$-simplex connecting $l_i$ to the next vertex in the path cannot be a part of either $L$ or $M$. If it were, then $L$ and $M$ would have a nonempty intersection, which contradicts our initial assumption that $K$ is not connected.

"$\Longleftarrow$": Now, let's consider the other direction. Suppose there are two points, namely $l_0$ and $m_0$, in $K$ that do not have a path connecting them. In this case, we can define $L$ as the path-connected subcomplex of $K$ that contains $l_0$, and $M$ as the path-connected subcomplex that contains $m_0$. If there exists a vertex $v_0$ in the intersection of $L$ and $M$ (i.e., $v_0 \in L \cap M \neq \emptyset$), then we can find a path from $l_0$ to $v_0$ and another path from $v_0$ to $m_0$. By concatenating these paths, we obtain a path from $l_0$ to $m_0$, which contradicts our initial assumption that there is no path between $l_0$ and $m_0$. Therefore, we conclude that $L$ and $M$ must have an empty intersection ($L\cap M= \emptyset$), indicating that $K$ is not connected.
\end{proof}

\begin{theorem}
Let $K_1, \ldots, K_p$ be the collection of all connected components of a complex $K$. Furthermore, let $H_{d_i}$ represent the $d$th homology group of $K_i$, and $H_d$ denote the $d$th homology group of $K$. In this context, we can establish that $H_d$ is isomorphic to the direct sum $H_{d_1} \oplus \cdots \oplus H_{d_p}$.
\end{theorem}

\begin{proof}
Let $L_d$ represent the group of $d$-chains of $K$, and $K_i$ denote the $i$th component of $K$. We can define $L_{d_i}$ as the group of $d$-chains of $K_i$. It is evident that $L_{d_i}$ is a subgroup of $L_d$. Furthermore, we observe that $L_d$ can be expressed as the direct sum of $L_{d_1}, \ldots, L_{d_p}$:
\begin{equation}
L_d = L_{d_1} \oplus \cdots \oplus L_{d_p}.
\end{equation}
Our goal is to demonstrate that a similar decomposition can be applied to the groups $B_d$ and $Z_d$. By considering $B_{d_i}$ as the image of $\partial$ restricted to the subgroup $L_{d_i}$, we can represent the group $B_d$ as the direct sum of these restrictions:
\begin{equation}
B_d = B_{d_1} \oplus \cdots \oplus B_{d_p}.
\end{equation}
Thus, for any element $c \in L_{d+1}$, which can be represented as:
\begin{align}
c = c_1 + \cdots + c_p, \quad \partial(c) = \partial c_1 + \cdots + \partial c_p \in B_d,
\end{align}
where $c_i \in L_{{d+1}_i}$.
Let us define $Z_{d_i}$ as the intersection of the kernel of $\partial$ and $L_{d_i}$. It follows that $Z_d$ can be expressed as the direct sum of $Z_{d_1}, \ldots, Z_{d_p}$:
\begin{equation}
Z_d = Z_{d_1} \oplus \cdots \oplus Z_{d_p}.
\end{equation}
To verify this, we observe that for an element $c \in L_d$ to belong to $Z_d$, we require $\partial(c) = 0$. However, we can express $\partial(c)$ as $\partial(c_1) + \cdots + \partial(c_p)$. Therefore, for $\partial(c) = 0$ to hold, it implies that $\partial(c_i) = 0$, indicating that $c_i \in Z_{d_i}$.
Since both $Z_d$ and $B_d$ can be decomposed componentwise, we can conclude that:
\begin{equation}
Z_d / B_d = Z_{d_1} / B_{d_1} \oplus \cdots \oplus Z_{d_p} / B_{d_p},
\end{equation}
and consequently:
\begin{equation}
H_d = H_{d_1} \oplus \cdots \oplus H_{d_p}.
\end{equation}
\end{proof}

\section{Singular Homology}
\section{Chain Complexes}
\section{Exact Sequences}
\section{Relative Homology Groups}
\section{The Equivalence of $H_k^\Delta$ and $H_k$}

\begin{thebibliography}{100}
\bibitem{1} Boissonnat, J. D., Chazal, F., Yvinec, M. (2018). Geometric and Topological Inference (Vol. 57). Cambridge University Press.
\bibitem{2} Edelsbrunner, H., Harer, J. L. (2022). Computational Topology: An Introduction. American Mathematical Society.
\bibitem{3} Hatcher, A. (2005). Algebraic Topology. Cambridge University Press.
\bibitem{4} Jonsson, J. (2011). Introduction to Simplicial Homology. Königliche Technische Hochschule. URL: \url{https://people.kth.se/~jakobj/doc/homology/homology.pdf}.
\bibitem{5} Khoury, M. (2022). Lecture 6: Introduction to Simplicial Homology. Topics in Computational Topology: An Algorithmic View. Ohio State University. URL: \url{http://web.cse.ohio-state.edu/~wang.1016/courses/788/Lecs/lec6-marc.pdf}.
\bibitem{6} Melodia, L., Lenz, R. (2021). Estimate of the Neural Network Dimension Using Algebraic Topology and Lie Theory. In Pattern Recognition. ICPR International Workshops and Challenges.
\bibitem{7} Nadathur, P. (2007). An Introduction to Homology. University of Chicago. URL: \url{https://www.math.uchicago.edu/~may/VIGRE/VIGRE2007/REUPapers/FINALFULL/Nadathur.pdf}.
\bibitem{8} Pontryagin L. S. (1952): Foundations of Combinatorial Topology. Graylock Press.
\end{thebibliography}
\end{document}