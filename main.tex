\documentclass{article}
\usepackage[utf8]{inputenc}
\usepackage[english]{babel}
\usepackage{amsthm,amssymb,amsmath}
\usepackage{titlesec}
\usepackage{url}
\usepackage[backend=biber]{biblatex}
\newtheorem*{theorem}{Theorem}
\newtheorem*{definition}{Definition}
\newtheorem*{lemma}{Lemma}

\newcommand{\NN}{\mathbb{N}}
\newcommand{\ZZ}{\mathbb{Z}}
\newcommand{\RR}{\mathbb{R}}
\newcommand{\QQ}{\mathbb{Q}}
\newcommand{\CC}{\mathbb{C}}

\title{\textbf{Simplicial Homology} \\[0.2cm] \large An introduction to simplicial homology theory}

\author{Luciano Melodia}
\date{\today}

\begin{document}

\maketitle
\begin{abstract}
This paper examines the fundamental ideas of simplicial structures that lead to simplicial homology theory and introduces singular homology to demonstrate the equivalence of homology groups of homomorphic topological spaces. It concludes with a proof of the equivalence of simplicial and singular homology groups.
\end{abstract}
\tableofcontents

\section{Simplicial Complexes}
We remark, that a set of points $X = \{x_0, x_1, \ldots, x_d\}$ in $\mathbb{R}^n$ is said to be \emph{affinely independent} if the points are not contained in any affine subspace of dimension less than $d$.

\begin{definition}
Given a set $X = \{x_0, x_1, \ldots, x_d\} \subset \mathbb{R}^n$ of $d+1$ \textbf{affinely independent} points, the $d$-dimensional simplex $\sigma$, or \textbf{\emph{$d$-simplex}}, spanned by $X$ is the set of convex combinations
\begin{equation}
	\sigma := \left\{\sum_{i=0}^{d} \lambda_i x_i \; \vert \; \sum_{i=0}^{d} \lambda_i = 1, \; \lambda_i \geq 0 \right\}.
\end{equation}
\end{definition}

By convention the empty set $\emptyset$ is added to the faces as the simplex spanned by the empty subset of the vertices. A $0$-simplex corresponds to a single point, a $1$-simplex corresponds to a line segment connecting two points, a $2$-simplex corresponds to a triangle, and a $3$-simplex corresponds to a tetrahedron. It is worth noting that the $d$-simplex is homeomorphic to the $d$-disk $D^d$. 

Additionally, $\sigma$ represents the convex hull of the points $X$, which is the smallest convex subset of $\mathbb{R}^n$ that contains $x_0, x_1, \ldots, x_d$. The \emph{faces} of the simplex $\sigma$ with vertex set $X$ are the simplices formed by subsets of $X$. An \emph{$d$-face} of a simplex is a subset of the vertices of the simplex with a cardinality of $d+1$. The faces of an $d$-simplex with a dimension less than $d$ are referred to as its \emph{proper faces}. Two simplices are considered to be \emph{properly situated} if their intersection is either empty or a face of both simplices. By identifying simplices along entire faces, we obtain the resulting \emph{simplicial complexes}.

\begin{definition}
A \textbf{simplicial complex} $K$ is a finite set of simplices such that
\begin{enumerate}
	\item For all simplices $\sigma \subseteq K$ with $\tau$ being a face of $\sigma$, it holds that $\tau \subset K$.
	\item $\sigma, \tau \subseteq K \implies \sigma, \tau$ are properly situated.
\end{enumerate}
\end{definition}

The \emph{dimension} of $K$ is the highest dimension of its simplices. For a simplicial complex $K$ in $\mathbb{R}^n$, its \emph{underlying space} $\vert K \vert \subset \mathbb{R}^n$ is the union of the simplices of $K$. The topology of $K$ is the topology induced on $\vert K \vert$ by the standard topology in $\mathbb{R}^n$. Notice that when its vertex set is known, a simplicial complex in $\mathbb{R}^n$ is fully characterized by the list of its simplices. Thus, we can describe it in pure combinatorial terms by \emph{abstract simplicial complexes}.

\begin{definition}
Let $V = \{v_1, \ldots, v_n\}$ be a finite set. An \textbf{abstract simplicial complex $\tilde{K}$} with vertex set $V$ is a set of finite subsets of $V$ satisfying the two conditions:
\begin{enumerate}
	\item The elements of $V$ belong to $\tilde{K}$.
	\item If $\sigma \subset \tilde{K}$ and $\tau \subseteq \sigma$, then $\tau \subset \tilde{K}$.
\end{enumerate}
\end{definition}

The abstract simplicial complex $\tilde{K}$ of a simplicial complex $K$ is also called its \emph{vertex scheme}. Conversely, if an abstract complex $\tilde{K}$ is the vertex scheme of a complex $K$ in $\mathbb{R}^n$, then $K$ is called a \emph{geometric realization} of $\tilde{K}$.

\begin{lemma}
Every finite abstract simplicial complex $\tilde{K}$ has a geometric realization in an Euclidean space.
\end{lemma}

\begin{proof}
Let $\{v_1,v_2, \ldots, v_n\}$ be the vertex set of $\tilde{K}$ where $n$ is the number of vertices of $\tilde{K}$, and let $\sigma \subset \mathbb{R}^n$ be the simplex spanned by $\{e_1, e_2, \ldots, e_n\}$ where, for any $i = 1, \ldots, n$, $e_i$ is the $i$th unit vector. Then $K$ is the subcomplex of $\sigma$ such that $[e_{i_0}, \ldots, e_{i_d}]$ is a $d$-simplex of $K$ if and only if $[v_{i_0}, \ldots, v_{i_k}]$ is a simplex of $\tilde{K}$.
\end{proof}

\paragraph{Note:} The realizations of an abstract simplicial complex are all homeomorphic to each other. The realization above is called \emph{natural realization}. It can also be proven that any finite abstract simplicial complex of dimension $n$ can be realized as a simplicial complex in $\mathbb{R}^{2d+1}$.

\section{Homology Groups}
Given the set $V$ of vertices of some simplex $\sigma$, we define an \emph{orientation} on the simplex by selecting an ordering of $V$. Vertex ordering that differ from our chosen order by an odd permutation are then designated as \emph{reversed}, while even permutations \emph{preserve} orientation. Thus, any simplex has only two possible orientations. An orientation on a $d$-simplex induces orientation on its $(d-1)$-faces. More formal, if $\sigma^{(d)} := (v_0, v_1, \ldots, v_d)$ is an oriented $d$-simplex, then the orientation of the $(d-1)$-face $\tau$ of $\sigma^{(d)}$ with vertex set $\{v_0,\ldots,v_{i-1},v_{i+1},\ldots,v_d\}$ is given by $\tau_i = (-1)^i (v_0, \ldots,v_{i-1},v_{i+1},\ldots,v_d)$.

\begin{definition}
Given a set $\{\sigma_1^{(d)}, \ldots, \sigma_k^{(d)}\}$ of arbitrarily oriented $d$-simplices of a complex $K$ and an abelian group $G$, we define a \textbf{$d$-chain} $c$ with coefficients $g_i \in G$ as a formal sum
\begin{equation}
c := g_1 \sigma^{(n)}_1 + g_2 \sigma^{(n)}_2 + \ldots + g_k \sigma^{(n)}_k = \sum_{i=1}^{k} g_i \sigma^{(n)}_i.
\end{equation}
\end{definition}

\paragraph{Note:} Henceforth we will assume that $G = (\mathbb{Z},+)$.

\begin{lemma}
The set of $d$-chains $L_d$ is an abelian group $(L_d,+)$.
\end{lemma}
\begin{proof}
The neutral element of the group is the empty chain $\sum_{i=1}^{d} e_G \sigma^{n}_i = e_G$. The sum of two chains is defined as $c+c' = \sum_{i=1}^{d} (g_i+g'_i) \sigma_i^{(n)}$, thus, we have that $c+c' \in L_d$. The associativity of the group operation of $L_d$ follows directly from the associativity of the group operation of $G$. The inverse element is also defined by the inverse element of the group $e_{L_d} = c + (-c) = \sum_{i=1}^{d} g_i \sigma_i^{(n)} - \sum_{i=1}^{d} (-g_i) \sigma_i^{(n)} =  \sum_{i=1}^{d} (g_i-g_i) \sigma_i^{(n)}$ with $c,-c \in L_d$.
\end{proof}

\begin{definition}

\end{definition}

\section{Singular Homology}
\section{Chain Complexes}
\section{Exact Sequences}
\section{Relative Homology Groups}
\section{The Equivalence of $H_k^\Delta$ and $H_k$}

\begin{thebibliography}{100}
\bibitem{1} Boissonnat, J. D., Chazal, F., Yvinec, M. (2018). Geometric and Topological Inference (Vol. 57). Cambridge University Press.
\bibitem{2} Edelsbrunner, H., Harer, J. L. (2022). Computational Topology: An Introduction. American Mathematical Society.
\bibitem{3} Hatcher, A. (2005). Algebraic Topology. Cambridge University Press.
\bibitem{4} Jonsson, J. (2011). Introduction to Simplicial Homology. Königliche Technische Hochschule. URL: \url{https://people.kth.se/~jakobj/doc/homology/homology.pdf}.
\bibitem{5} Khoury, M. (2022). Lecture 6: Introduction to Simplicial Homology. Topics in Computational Topology: An Algorithmic View. Ohio State University. URL: \url{http://web.cse.ohio-state.edu/~wang.1016/courses/788/Lecs/lec6-marc.pdf}.
\bibitem{6} Nadathur, P. (2007). An Introduction to Homology. University of Chicago. URL: \url{https://www.math.uchicago.edu/~may/VIGRE/VIGRE2007/REUPapers/FINALFULL/Nadathur.pdf}.
\bibitem{7} Melodia, L., Lenz, R. (2021). Estimate of the Neural Network Dimension Using Algebraic Topology and Lie Theory. In Pattern Recognition. ICPR International Workshops and Challenges.
\end{thebibliography}
\end{document}
